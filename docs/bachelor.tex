\input{bachelor-praeambel.tex} % Importiere die Einstellungen aus der Präambel
% hier beginnt der eigentliche Inhalt
\begin{document}
\pagenumbering{Roman} % große Römische Seitenummerierung
\pagestyle{empty}

% Titelseite
\clearscrheadings\clearscrplain

\begin{center}
\begin{Huge}
Institut für Mathematik und Informatik\\
\vspace{3mm}
\end{Huge}{\Large Fernuniversiät Hagen}\\

\vspace{28mm}
\begin{Large}
Vergleichende Implementierung und Evaluierung einer Echtzeitvisualisierung von geographischen Schiffsbewegungsdaten in HTML5\\
\end{Large}
\vspace{8mm}
Bachelorarbeit\\
\vspace{0.4cm}
\vspace{5 cm}
Barbara Drüke \\
Matrikel-Nummer 7397860\\
\vspace{3cm}

{\bf Abgabedatum:} 23.05.2013\\
\vspace{3cm}

\begin{tabular}{ll}
{\bf Betreuer} & Dr. Jörg Brunsmann\\
{\bf Erstprüfer}&Prof. Dr. Hemmje\\
{\bf Zweitprüfer}&Dr. Jörg Brunsmann\\
\end{tabular}

\end{center}
\clearpage


\pagestyle{useheadings} % normale Kopf- und Fußzeilen für den Rest

\tableofcontents
\listoffigures
\listoftables
\lstlistoflistings





% richtiger Inhalt
%---------------------------------------------------------------------------------------------------------------------------------------------
\chapter{Einleitung}
\pagenumbering{arabic} % ab jetzt die normale arabische Nummerierung

Die Vesseltracker.com GmbH ist ein Schiffsmonitoring und -reporting-Dienstleister. Der kostenpflichtige Dienst stellt den Kunden umfangreiche Informationen zu Schiffen weltweit zur Verfügung. Dabei handelt es sich einerseits um Schiffs-Stammdaten und andererseits um Schiffs-Postionsdaten. Die Positionsdaten sind AIS (Automatic Identification System) -Daten, wie sie von allen Schiffen über Funk regelmäßig zu senden sind.

\begin{wrapfigure}{r}{0.6\textwidth}
  \begin{center}
    \includegraphics[width=0.58\textwidth]{images/Exposee_graphik_Webapp}
  \end{center}
  \caption{Architektur der Web-Anwendung der Vesseltracker.com-GmbH}
\end{wrapfigure}

Vesseltracker.com unterhält ein Netzwerk von ca. 800 terrestrischen AIS-Antennen, mit denen küstennahe AIS-Meldungen empfangen und via Internet an einen zentralen Rohdatenserver geschickt werden. Der Rohdatenserver verarbeitet die Meldungen und gibt sie umgewandelt und gefiltert an die Anwendungen des Unternehmens weiter.
Zusätzlich erhält das Unternehmen AIS-Daten via Satellit über einen Kooperationspartner. Damit werden die küstenfernen Meeresgebiete und Gegenden, in denen Vesseltracker.com keine AIS-Antenne betreibt, abgedeckt.
Die Kernanwendung des Unternehmens ist eine Webanwendung, die die terrestrischen AIS-Daten in einer Geo-Datenbank speichert und sie mit den Schiffs-Stammdaten und Satelliten-AIS-Daten in Beziehung setzt.

Für eine geographische Visualisierung der Schiffspositionen existiert das sogenannte 'Cockpit', wo die Schiffe als Icons auf Openstreetmap-Karten dargestellt werden. Diese Karte zeigt jeweils alle Schiffe an, die sich in dem frei wählbaren Kartenausschnitt zu der Zeit befinden. Aktualisiert werden die Positionsinformationen jeweils bei Änderung des betrachteten Bereichs oder einmal pro Minute. Detailinformationen erhält der Nutzer durch ein Click-Popup über das Icon des Schiffes. Darüber kann er sich auch die gefahrene Route der letzten 24 Stunden anzeigen lassen.


\begin{figure}[H]
  \centering
  \includegraphics[width=6in]{images/Cockpit_Elbe}
  \caption[Ansicht der Elbe hinter Hamburg in der ‘Cockpit’-Anwendung]{Ansicht der Elbe hinter Hamburg in der ‘Cockpit’-Anwendung}
\end{figure}

\section{Motivation für diese Arbeit}\label{s.Motivation für diese Arbeit}

Aus mehreren Gründen entstand der Plan, dem Portfolio des Unternehmens neben der existierenden Cockpit-Anwendung eine Real-Time-Darstellung der geographischen Schiffspositionen hinzuzufügen.
\begin{itemize}

\item Aufgrund der herausragenden Qualität des vesseltracker.com Antennen-Netzwerks sind die verfügbaren terrestrischen AIS-Daten höchst aktuell, aktualisieren sich kontinuierlich und erreichen eine hohe weltweite Abdeckung des Schiffsverkehrs. Damit ist es möglich, die Situation des Schiffsverkehrs in beliebigen Häfen, Wasserstraßen und Küstengebiete weltweit und sekundengenau zu präsentieren. Diese Genauigkeit wird in der Cockpit-Anwendung nicht vollständig genutzt.

\item Real-Time-Anwendungen gewinnen laufend an Bedeutung. Ihre Verbreitung wird durch den Fortschritt der verfügbaren Webtechnologien auf breiter Basis unterstützt. Mitbewerber auf dem Markt für AIS-Daten (z.B. Fleetmon.com) bieten bereits Echtzeit-Darstellungen ihrer AIS-Daten an. Um in diesem Geschäftsfeld weiterhin eine Spitzenposition beanspruchen zu können, sollte auch Vesseltracker eine Real-Time-Anwendung zur Schiffsverfolgung für Desktop-Computer zur Verfügung stellen und in einem nächsten Schritt auch für Mobile Devices.
\item Ein Phänomen in der menschlichen Wahrnehmung liefert ein weiteres Argument, die Cockpit-Anwendung durch eine Real-Time-Anwendung zu ergänzen oder sogar abzulösen: Aufgrund der sogenannten Veränderungsblindheit oder “Change Blindness” werden Veränderungen an einem Objekt (in diesem Fall die Position eines Schiffs-Icons auf der Karte) in der Wahrnehmung überdeckt, wenn im selben Augenblick Veränderungen an der Gesamtsicht vonstatten gehen \cite{changeblindness}. Genau dies geschieht im Cockpit, wenn nach dem Laden neuer Positionsdaten alle Schiffs-Icons neu gerendert und Namens-Fähnchen gelöscht oder hinzugefügt werden. Dieses kurze “Flackern” macht es dem Betrachter fast unmöglich, die Positionsänderung eines Schiffes auf der Karte mit dem Auge zu verfolgen.
\end{itemize}


\section{Aufbau der Arbeit}\label{s.Aufbau der Arbeit}
Im Kapitel \ref{c.Realtime-Schiffsverfolgung per AIS-Daten-Strom} werden mögliche Anwendungs-Szenarien genauer beleuchtet und die funktionalen und nicht funktionalen Anforderungen an die geplante Anwendung herausgestellt. Anschließend wird die Systemarchitektur der geplanten Anwendung grob entworfen.
In Kapitel \ref{s.Grundlagen} wird kurz auf die technischen Grundlagen eingegangen: die AIS-Technologie, die bei der Gewinnung des Datenmaterials verwendet wird und damit für Art und Format der Daten verantwortlich ist; das Javascript-Framework node.js sowie Google Dart, die bei der Programmierung der Anwendung zum Einsatz kommen; HTML5-Websockets, weil sie für die Verteilung der Daten eingesetzt werden; das OpenStreetMap-Projekt als Lieferant des Kartenmaterials wird kurz vorgestellt, sowie die Leaflet-Bibliotheken, mit deren Hilfe die Schiffsobjekte auf die Karte gerendert werden.

In Kapitel \ref{s.Implementierungen} werden zunächst die Gründe für die spezifische Auswahl an Implementierungen dargelegt. Anschließend wird die Vorgehensweise bei der Implementierung erläutert und zwar zunächst ausführlich für die jeweils erste Server- bzw. Client-Implementierung (socket.io-Server \ref{socket.io-Server} und socket.io-Client \ref{socket.io-Client}). Anschließend werden für die alternativen Implementierungen nur jeweils die Unterschiede herausgestellt. Die fertigen Programme werden in Kapitel \ref{Ergebnisse} getestet und nach verschiedenen Aspekten verglichen.
Kapitel \ref{Fazit} fasst die Ergebnisse zusammen und gibt einen Ausblick auf mögliche Weiterentwicklungen.

%---------------------------------------------------------------------------------------------------------------------------------------------

\chapter{Real-Time-Schiffsverfolgung per AIS-Daten-Strom}\label{c.Realtime-Schiffsverfolgung per AIS-Daten-Strom}

\section{ Anwendungsfälle}\label{s.Anwendungsfälle}

Hafendienstleister wie Schlepper, Lotsen oder Festmacher verschaffen sich über einen Monitor einen Überblick über die Arbeitsvorgänge in ihrem jeweiligen Heimathafen, z.B. welche Schlepper schleppen welches Schiff, wo gehen Lotsen an oder von Bord, welche Tanker betanken welche Schiffe, usw. Sie kontrollieren die Ausführung der eigenen Aufträge oder auch die der Mitbewerber.
Die Anwendung läuft hierbei eigenständig, das heißt, Zoomstufe und Kartenausschnitt ändern sich nicht oder nur selten. Es ist also notwendig, dass die Anwendung unabhängig von Benutzer-Interaktionen immer die aktuellsten verfügbaren Daten anzeigt.\\
Ein verwandter Anwendungsfall betrifft Nutzer, für die die Beobachtung, bzw. Überwachung bestimmter Wasserverkehrswege oder Häfen von besonderem Interesse ist. Dies trifft auf Sicherheitsorgane (z.B. die Wasserschutzpolizei), Schiffsfotografen und Nutzer der Passagierschifffahrt zu.\\
Reedereien beobachten das Einlaufen, Anlegen, Festmachen oder Ablegen und Auslaufen ihrer Schiffe in entfernten Häfen, wo es keine Unternehmensniederlassung gibt. Zum Beispiel kontrollieren sie, wann und an welchen Liegeplätzen ein Schiff wie lange festmacht.
Dazu ist es zum einen notwendig, auf eine geringe Zoomstufe heraus- und auf einen anderen Hafen wieder hineinzoomen zu können. Zum anderen soll die Anwendung Schnittstellen bieten, um zusätzliche Informationen aus dem vesseltracker.com Datenpool (z.B. Liegeplatzinformationen) anzufordern.\\
Die vesseltracker.com GmbH nutzt die Real-Time-Anwendung, um die vom Unternehmen angebotenen Daten zu präsentieren und zu bewerben. Dabei ist es wichtig, dass die Anwendung gesendete AIS-Signale im Schnitt in weniger als einer Sekunde auf dem Monitor als Position oder Positionsänderung darstellen kann und dass die Schiffsbewegungen fließend ohne  “Flackern” dargestellt werden. Damit kann vesseltracker.com die größere Genauigkeit und Aktualität der eigenen Daten gegenüber denen anderer Anbieter herausstellen.

Die Anwendungsfälle verdeutlichen noch einmal, dass der zusätzliche Nutzen der Real-Time-Anwendung gegenüber der Cockpit-Anwendung nicht ausschließlich in der höheren Aktualität liegt, denn die Daten im Cockpit sind ja ebenfalls im Minutenbereich aktuell. Ein wichtiger Vorteil liegt vielmehr in der Lebendigkeit der Darstellung. Bewegte Objekte binden stärker die Aufmerksamkeit des Betrachters. Sie sind ohne Anstrengung mit dem Auge zu verfolgen und geben der Anwendung einen gewissen Unterhaltungswert.
\newpage
\section{Beschreibung der Anforderungen}\label{s.Beschreibung der Anforderungen}


\subsection{Funktionale Anforderungen}\label{Funktionale Anforderungen}
\begin{itemize}

\item als Datenquelle sollen ausschließlich die vom Rohdatenserver als JSON-Datenstrom zur Verfügung gestellten AIS-Informationen dienen
\item Schiffe sollen an ihrer aktuellen (Real-Time-) Position auf einer Karte im Browser dargestellt werden
\item Positionsänderungen einzelner Schiffe sollen ad hoc sichtbar gemacht werden
\item die Schiffsbewegungen auf der Karten sollen nicht sprunghaft, sondern fließend erscheinen (Animation der Schiffsbewegungen in dem Zeitraum zwischen zwei Positionsmeldungen)
\item die Karte soll in 16 Zoomstufen die Maßstäbe von 1:2000 bis 1: 200 Mio abdecken
\item Schiffe sollen auf der Karte als Symbole dargestellt werden, die den Navigationsstatus und gegebenenfalls den Kurs wiederspiegeln
\item bei hoher Auflösung / großem Zoom-Level und ausreichend statischen AIS-Informationen soll ein Schiff als maßstabsgetreues Polygon (genauer: Fünfeck) in die Karte gezeichnet werden.
\item bei geringer Auflösung / niedrigem Zoom-Level soll ein Eindruck über die Verteilung der empfangenen Schiffe vermittelt werden, ohne jedoch jedes Schiff tatsächlich darzustellen
\item Detail-Informationen zu jedem Schiff sollen als Popups über das Symbol auf der Karte abrufbar sein
\end{itemize}

\subsection{Nicht funktionale Anforderungen}\label{Nicht funktionale Anforderungen}
\begin{itemize}
\item die von den Antennen empfangenen AIS-Daten sind mit minimaler Verzögerung (unter 500 msec) auf der Karte darzustellen
\item die Anwendung soll ca. 300 gleichzeitige Client-Verbindungen erlauben und skalierbar sein
\item als Clients der Anwendung sollen die gängigen Browser in den am meisten verbreiteten Versionen unterstützt werden (Microsoft Internet Explorer, Google Chrome, Moziila Firefox, Apple Safari) 
\item der Programm-Code wird auf Github als privates repository gehostet
\item verwendete Software-Module sollen frei zugänglich (open source) sein 
\item als Kartenmaterial sind die von vesseltracker gehosteten OpenstreetMap-Karten zu verwenden
\item ein Prototyp der Anwendung soll zeitnah zur Verfügung stehen, um Mitarbeitern und Kunden zu ermöglichen, ihre Anforderungen genauer zu spezifizieren oder weitere Anforderungen zu formulieren.
\end{itemize}

\section{Grobentwurf der Anwendung}\label{s.Grobentwurf der Anwendung}

\begin{wrapfigure}{r}{0.6\textwidth}
  \begin{center}
    \includegraphics[width=0.6\textwidth]{images/Exposee_graphik_Realtimeapp}
  \end{center}
  \caption{Architektur-Entwurf der Real-Time Web-Anwendung}
\end{wrapfigure}
Die eingehende Schnittstelle der zu erstellenden Anwendung ist die Verbindung zum Rohdatenserver, die als TCP-Verbindung ausgeführt ist und einen JSON-Datenstrom liefert.
Die ausgehende Schnittstelle ist der HTTP-Client (Browser).
Zu erstellen ist also eine Client-Server-Anwendung, in der der Server zweierlei zu leisten hat, nämlich 
\begin{enumerate}
 \item eine TCP-Socket-Verbindung zum Rohdatenserver zur Abfrage des AIS-Daten-Stroms zu unterhalten und
  \item eine bidirektionale Verbindungen zu vielen HTTP-Clients herzustellen, in der die Clients Änderungen des betrachteten Kartenausschnittes an den Server senden können und der Server jederzeit den Client über relevante, aus dem JSON-Datenstrom ausgelesene, Schiffsbewegungen im betrachteten Kartenausschnitt informieren kann.
\end{enumerate}


%---------------------------------------------------------------------------------------------------------------------------------------------
\chapter{Grundlagen}\label{s.Grundlagen}
\section{Automatisches Informationssystem}\label{s.Automatisches Informationssystem (AIS)}
Das Automatic Identification System (AIS) ist ein UKW-Funksystem im Schiffsverkehr, das seit 2004 für alle Berufsschiffe über 300 BRZ ? in internationaler Fahrt und seit 2008 auch für solche über 500 BRZ in nationaler Fahrt verpflichtend eingeführt worden ist. Es soll dabei helfen, Kollisionen zwischen Schiffen zu verhüten und die landseitige Überwachung und Lenkung des Schiffsverkehrs zu erleichtern. Außerdem verbessert AIS die Planung an Bord, weil nicht nur Position, Kurs und Geschwindigkeit der umgebenden Schiffe übertragen werden, sondern auch Schiffsdaten (Schiffsname, MMSI-Nummer, Funkrufzeichen, etc.). AIS ist mit UKW-Signalen unabhängig von optischer Sicht und Radarwellenausbreitung \cite{wiki:ais}.\\
Für die Nutzung von AIS ist ein aktives, technisch funktionsfähiges Gerät an Bord Voraussetzung, das sowohl Daten empfängt als auch Daten sendet. Für Schiffe der Berufsschifffahrt sind Klasse-A-Transceiver an Bord vorgesehen, für nicht ausrüstungspflichtige Schiffe genügen Klasse-B-Transceiver, die mit niedriger VHF-Signalstärke und weniger häufig senden.  \\
Die dynamischen Schiffsdaten (s.u.) erhält der AIS-Transceiver vom integrierten GPS-Empfänger, bei Klasse A auch von der Navigationsanlage des Schiffes. Die Vorausrichtung (Heading) kann über eine NMEA-183-Schnittstelle vom Kompass eingespeist werden.

Die AIS-Einheit sendet schiffsspezifische Daten, die von jedem AIS-Empfangsgerät in Reichweite empfangen und ausgewertet werden können:
\subsubsection{Statische Schiffsdaten} \label{Statische Schiffsdaten}
\begin{itemize}
\item IMO-Nummer\footnote{http://de.wikipedia.org/wiki/Schiffsnummer\#IMO-Nummer}
\item Schiffsname
\item Rufzeichen\footnote{http://de.wikipedia.org/wiki/Schiffsnummer\#Unterscheidungssignal}
\item MMSI-Nummer\footnote{http://de.wikipedia.org/wiki/Schiffsnummer\#MMSI}
\item Schiffstyp (Frachter, Tanker, Schlepper, Passagierschiff, SAR, Sportboot u. a.)
\item Abmessungen des Schiffes (Abstand der GPS-Antenne von Bug, Heck, Backbord- und Steuerbordseite)
\end{itemize}

\subsubsection{Dynamische Schiffsdaten} \label{Dynamische Schiffsdaten}
\begin{itemize}
\item Navigationsstatus (unter Maschine, unter Segeln, vor Anker, festgemacht, manövrierunfähig u. a.)
\item Schiffsposition (LAT, LON in WGS 84)
\item Zeit der Schiffsposition (nur Sekunden)
\item Kurs über Grund (COG)
\item Geschwindigkeit über Grund (SOG)
\item Vorausrichtung (HDG)
\item Kursänderungsrate (ROT)
\end{itemize}

\subsubsection{Reisedaten} \label{Reisedaten}
\begin{itemize}
\item aktueller maximaler statischer Tiefgang in dm
\item Gefahrgutklasse der Ladung (IMO)
\item Reiseziel (UN/LOCODE)\footnote{http://www.unece.org/cefact/locode/service/location.html}
\item geschätzte Ankunftszeit (estimated Time of Arrival = ETA)
\item Personen an Bord
\end{itemize}

Der Navigationsstatus und die Reisedaten müssen vom Wachoffizier manuell aktualisiert werden. Gesendet werden die AIS-Signale auf zwei UKW-Seefunkkanälen (Frequenzen 161,975 MHz und 162,025 MHz), wobei die Sendeintervalle abhängig sind von der Klasse, dem Manöverstatus und der Geschwindigkeit.

\begin{table}[!hbt]\vspace{1ex}\centering
\small
\texttt{
\begin{tabular}{|l|l|l|l|}\hline
Klasse &Manöver-Status & Geschwindigkeit &Sendeintervall\\\hline\hline
Class A&geankert/festgemacht&<3kn&3 min\\
Class A&geankert/festgemacht&>3kn&10 sec\\
Class A&in Fahrt&0-14kn&10 sec\\
Class A&in Fahrt, Kursänderung&0-14&3 1/3 sec\\
Class A&in Fahrt&14-23kn&6 sec\\
Class A&in Fahrt, Kursänderung&14-23&2 sec\\
Class A&in Fahrt&>23kn&2 sec\\
Class B&&<2 kn&3 min\\
Class B&&>2 kn&30 sec\\\hline
\end{tabular}
}
\caption[Intervalle, in denen Schiffe AIS-Nachrichten aussenden] {Intervalle, in denen Schiffe AIS-Nachrichten aussenden}
\end{table}

Für AIS-Daten sind 22 standardisierte Nachrichtentypen bzw. Telegramme festgelegt. Für diese Arbeit interessieren nur die regulären Positionsmeldungen (dynamische Schiffsdaten) der Klasse-A-Transceiver (Typ 1- ,2-  und 3-Messages) und die regulären Meldungen von (statischen) Schiffs- und Reisedaten der Klasse-A-Transceiver (Typ 5-Messages). 
\section{OpenStreetMap}\label{OpenStreetMap}
OpenStreetMap\footnote{http://www.openstreetmap.org/} ist eine freie, editierbare Karte der gesamten Welt auf der Basis von Daten, die von einer breiten Nutzergemeinde zusammengetragen werden. Inzwischen kann die Qualität der Karten mit denen proprietärer Angebote mithalten und übertrifft sie sogar in manchen Bereichen. Die Daten können gemäß der entsprechenden Freien Lizenz frei heruntergeladen und genutzt werden unter der Bedingung, dass sie nur unter der “Creative Commons Attribution-Share-Alike” (CC-BY-SA) -Lizenz weitergegeben werden.

\section{Leaflet}\label{Leaflet}
Leaflet ist eine open source JavaScript-Bibliothek\footnote{http://leafletjs.com/reference.html} für interaktive Karten, die von einer Gruppe um Vladimir Agafonkin geschrieben wurde. Die Bibliothek zeichnet sich im Vergleich zu OpenLayers durch ein klares, schlankes Design aus und überzeugt durch gute Performance. Leaflet unterstützt alle Plattformen auch im mobilen Bereich mithilfe von HTML5 und CSS3. Es ist hinreichend dokumentiert und verfügt über gut lesbaren Quellcode.

\section{Bidirektionale Kommunikation über HTML5 Websockets}\label{s.Websockets}
In der Entwicklung der Kommunikationstechnologien im Internet galt lange Zeit das request/response Paradigma, nach dem Anfragen eines Clients von einem Server beantwortet werden. Dieses Paradigma wird Stück für Stück aufgebrochen durch kontinuierliche Weiterentwicklungen in Richtung einer bidirektionalen Kommunikation zwischen Server und Client.\\
Schon seit HTTP Long Polling, HTTP Streaming und Ajax on demand ist es für Serveranwendungen möglich, nach einem initialen Verbindungsaufbau durch den Client, beim serverseitigen Eintreffen neuer Daten scheinbar selbständig einen Datenaustausch zum Client zu initieren. Dabei handelt es sich eigentlich nur um die aufgeschobene Beantwortung einer zuvor gestellten Client-Anfrage.\\
Der Nachteil dieser Technologien liegt darin, dass sie, weil sie Nachrichten über das HTTP-Protokoll austauschen, einen großen Überhang an Header-Informationen mitzusenden gezwungen sind, der sich in Summe negativ auf die Latenzzeit auswirkt \cite{varaksin}. Damit sind diese Technologien für zeitkritische (Real-Time-) Anwendungen nicht unbedingt geeignet.
\\
Das 2011 eingeführte Websocket-Protokoll dagegen spezifiziert eine API (HTML5-Websocket API-Spezifikation \footnote{http://www.w3.org/TR/2011/WD-websockets-20110419/}), die eine echte bidirektionale Socket-Verbindung zwischen Server und Client ermöglicht, in der beide Seiten jederzeit Daten schicken können. Dieser Socket wird im Anschluss an einen intialen HTTP-handshake aufgebaut, indem Server und Client  einen Upgrade der Verbindung auf das Websocket-Protokoll aushandeln  \cite{html5rocks}. 

\section{node.js}\label{node.js}
Node.js\footnote{http://www.nodejs.org/} ist ein Framework zur Entwicklung serverseitiger Webanwendungen in Javascript. Es wurde 2009 von Ryald Dahl veröffentlich und hat seitdem viel Aufmerksamkeit erregt, weil Anwendungen in node.js
\begin{itemize}

\item hoch performant
\item skalierbar
\item und echtzeitfähig sind.
\end{itemize}

Diese Eigenschaften sind größtenteils dem Konzept des asynchronen, nicht blockierenden I/O von javascript im Allgemeinen und node.js im Besonderen geschuldet.
Javascript ist von Anfang asynchron konzipiert für die Verwendung im Webbrowser, wo synchrone Verarbeitung wegen der Verzögerung des Seitendarstellung nicht in Frage kommt. Den gleichen Ansatz übernimmt node.js für die Serverseite.
\\
Node.js arbeitet single-threaded und eventbasiert. Die zentrale Kontrollstruktur, die den Programmablauf steuert, ist der Event-Loop. Er empfängt Events, die von Programm- oder Nutzeraktionen ausgelöst werden und setzt sie in Callback-Funktionen um.
Kommt es im Programmablauf zur Interaktion mit einer externen Ressource, wird diese Interaktion in einen neuen Prozess ausgelagert und mit einer Callback-Methode versehen. Anschließend kann der Event Loop weitere aufgelaufene Events verarbeiten. Ist die Interaktion abgeschlossen bekommt der Event Loop ein Signal und setzt beizeiten die Verabeitung mit der entsprechenden Callback-Methode  fort \cite{teixeira}.\\
Node.js bringt als Laufzeitumgebung die V8-Javascript-Engine mit, die die Ausführung von Javascript-Code durch Just-In-Time-Kompilierung optimiert. Außerdem bietet node.js eine direkte Unterstützung für das HTML5-Websocket-Protokoll. Mit der Unterstützung des JSON-Datenformats sind alle notwendigen Bausteine zusammen für skalierbare, echtzeitfähige Serveranwendungen.
Außerdem lassen sich mit dem node.js Package Manager npm jederzeit weitere Pakete aus dem wachsenden Angebot nachinstallieren und verwalten.\\
Als konkrete Pakete für Websockets standen innerhalb von node.js zum Zeitpunkt der Implementierung (November 2012) die Bibliotheken websocket\footnote{https://github.com/Worlize/WebSocket-Node} und socket.io\footnote{http://socket.io} zur Verfügung. Die Bibliothek websocket genügt der HTML5-Websocket-Api-Spezifikation (s.o). Socket.io erweitert die Funktionalität des websockets um heartbeats, timeouts and disconnection support. Außerdem kapselt socket.io die Details des Nachrichtenaustauschs: Bei Browsern, die Websockets nicht unterstützen, handelt socket.io die bestmögliche Verbindungsalternative aus in der Reihenfolge: 
-->    WebSocket 
 -->   Adobe® Flash® Socket
  -->  AJAX long polling
   --> AJAX multipart streaming
 -->   Forever Iframe
 -->   JSONP Polling\footnote{http://socket.io/\#browser-support}.
Um socket.io zu nutzen können, muss im Browser-Client die Datei socket.io.js geladen werden.
\section{Google Dart}\label{s.Google Dart }

\subsubsection{Motivation für Dart}
Dart ist eine von der Firma Google als open source Projekt seit ca. 2 Jahren explizit für Webanwendungen entwickelte Programmiersprache. Das Ziel ist es, eine Sprache zu entwickeln, die komplexe Webanwendungen besser unterstützt als Javascript mit seinen historisch bedingten Mängeln und Schwächen.
Das Entwicklerteam definiert die Design-Ziele folgendermaßen\footnote{http://www.dartlang.org/docs/technical-overview/\#goals}:
Dart soll
\begin{itemize}   
\item eine sowohl strukturierte als auch flexible Web-Programmiersprache sein
\item sich für Programmierer vertraut anfühlen und intuitiv erlernbar sein 
\item mit seinen Sprachkonstrukten performant sein und schnell zur Ausführung kommen
\item auf allen Webdevices wie Mobiles, Tablets, Laptops und Servern gleichermaßen lauffähig sein
\item alle gängigen Browser unterstützen.
\end{itemize}

\subsubsection{Spracheigenschaften von Dart}\label{s.Spracheigenschaften von Dart}
\begin{itemize}

\item Dart arbeitet {\bf ereignisbasiert} und {\bf asynchron} und in einem einzigen Thread ganz nach dem Vorbild von node.js.
\item  In Browsern mit der{\bf Dart-Virtual-machine} (z.Zt. nur Google Dartium) kann nativer Dart-Code ausgeführt werden. In allen anderen Browsern wird der Dart-Code zu Javascript kompiliert. Dazu muss im Browser-Client nur die Datei dart.js aus dem Dart-Package browser geladen werden.
 
\item {\bf Klassen } sind ein wohlbekanntes Sprachkonzept zur Kapselung und Wiederverwendung von Methoden und Daten. Jede Klassen definiert implizit ein Interface.
\item {\bf Optionale Typisierung:}
Die Typisierung in Dart ist optional, das heißt sie führt nicht zu Laufzeitfehlern. Sie ist als Werkzeug für den Entwickler gedacht, zur besseren Verständlichkeit des Codes und als Hilfe beim Debuggen.
\item Die  {\bf Gültigkeitsbereiche}
von Variablen in Dart gehorchen einfachen, intuitiv nachvollziehbaren Regeln: Variablen sind gültig in dem Block ({...}), in dem sie definiert sind.
\item
Zur {\bf Parallelverarbeitung} nutzt Dart das Konzept von Isolates (übernommen von ERLANG), eine Art Leightweigth Processes. Isolates greifen nicht auf einen gemeinsamen Speicherbereich zu und teilen nicht denselben Prozessor-Thread. Isolates kommunizieren miteinander ausschließlich über Nachrichten (über SendPort und ReceivePort). Sie werden gesteuert von einem übergeordneten Event Loop.
\item Der {\bf DartEditor} ist eine Entwicklungsumgebung für die Entwicklung von Dart Web- und Serverapplikationen. Sie beinhaltet das {\bf Dart SDK } und den {\bf Dartium Browser} mit der {\bf Dart VM}.
\item Der {\bf dart2js Compiler} ist ebenfalls im DartEditor enthalten und kompiliert Dart-Code zu Javascript-Code, der für die Chrome V8 Javascript engine optimiert ist.
\item Mit {\bf Pub} verfügt Dart über einen Package Manager vergleichbar dem node.js Package Manager npm.
\end{itemize}
\cite{dartvsjs}\cite{builddartapps}

\subsubsection{Einbindung von Javascript-Bibliotheken in Dart mit js-interop}\label{js-interop}
Für die Verwendung von Javascript-Bibliotheken in Dart-Code existiert die Dart-Bibliothek {\bf js-interop}\footnote{https://github.com/dart-lang/js-interop}. Damit können Dart-Anwendungen Javascript-Bibliotheken verwenden und zwar sowohl in nativem Dart, das in der Dart-Virtual-Machine ausgeführt wird als auch in zu Javascript kompiliertem Dart-Code.\\

Wenn das Dart-Package js in eine Dart-Anwendung eingebunden ist, kann ein sogenannter {\bf Proxy} zum javascript-Kontext der Seite erstellt werden. Referenzen an diesen Proxy werden automatisch in den Javascript-Kontext umgeleitet. Auf oberster Ebene lassen sich damit Javascript-Arrays und -Maps generieren, die mit den entsprechenden Objekten in Dart korrespondieren. Über diesen Proxy können aber auch Proxies zu beliebigen Javascript-Objekten erstellt werden, deren Eigenschaften und Methoden im Javascript-Kontext zur Verfügung stehen.\\

Um Dart-Funktionen aus dem Javascript-Kontext heraus aufzurufen, wird die entsprechende Funktion in ein {\bf Callback-Objekt} geladen, das entweder ein einziges Mal (js.Callback.once(dart function)) oder mehrmals (js.Callback.many(dart function)) aufrufbar ist. Um die Lebensdauer dieser Proxies und Callback-Objekte zu verwalten, benutzt Dart das Scope-Konzept: Per default haben alle Proxies nur lokale Gültigkeit. Sollen sie den Ausführungszeitraum des Scopes überdauern, können sie ausdrücklich aufbewahrt werden (js.retain(js.Proxy-Object)), müssen dann aber zu Vermeidung von memory leaks auch explizit wieder freigegeben werden (js.release(js.Proxy-Object)). Dasselbe gilt für Callback-Objekte, die mehrmals aufrufbar sind \cite{js-interop}.

\subsubsection{Dart-Websockets}
Für serverseitiges Dart, das auf der serverseitigen Dart-VM läuft. existiert das Paket Dart:io. Es ermöglicht Zugriff auf das Dateisystem und auf Prozesse. In Dart:io existiert auch eine Websocket-Implementierung,\footnote{http://api.dartlang.org/docs/releases/latest/dart\_io/WebSocket.html} mit der bereits einfache Websocket-Server geschrieben werden können\footnote{http://www.dartlang.org/docs/dart-up-and-running/contents/ch05.html}.


%---------------------------------------------------------------------------------------------------------------------------------------------


\chapter{Implementierungen}\label{s.Implementierungen}

\section{Strategie bei der Vorgehensweise}\label{Strategie bei der Vorgehensweise}
Zunächst wird eine Implementierung gewählt, die die besten Chancen hat, alle funktionalen und nicht funktionalen Anforderungen zu erfüllen. Dies ist eine Lösung in Javascript mit dem node.js-Framework und dem socket.io-Websocket (Abschnitt \ref{socket.io-Server} und \ref{socket.io-Client}). Node.js-Serveranwendungen werden schon länger mit guten Ergebnissen in Netzwerken eingesetzt, besonders für Real-Time-Anwendungen mit vielen gleichzeitig verbunden Clients. Das socket.io-Paket wird genutzt, weil durch die Kapselung der verschiedenen Transportmechanismen die Bedienung einer maximalen Anzahl verschiedener Browser, auch in älteren Versionen möglich ist, ohne den Implementierungsaufwand unverhältmismäßig zu erhöhen. Die Implementierung dieser Lösung steht im Zeitplan vorn, um der Anforderung von Unternehmensseite nach einer zeitnahen Umsetzung und Auslieferung zu entsprechen.
In einem zweiten Schritt wird eine vergleichbare Implementierung in Google Dart ausgeführt. Ein Jahr nach der offiziellen Vorstellung (Oktober 2011) befindet sich Dart inzwischen im Zustand
„Technology Preview", so dass ein Einsatz clientseitig im Testbetrieb möglich sein sollte. Der zweite Beta-Release fand im Dezember 2012 statt. Ein dritter Beta-Release ist angekündigt. Obwohl ein ausschließlicher Einsatz von Dart im Produktivsystem noch nicht möglich ist, soll Dart als mögliche Alternative zu Javascript für komplexe Webanwendungen begutachtet werden. Die Lösung in Dart testet die Fähigkeiten und Möglichkeiten, die Dart im Vergleich zu Javascript hat und bietet, um eine datenintensive Real-Time-Anwenundung umzusetzen.
\section{Notwendige Strategie-Korrekturen}\label{Strategie-Korrektur}
Der ursprüngliche Plan, sowohl einen Server als auch einen Client in Dart zu schreiben, musste fallen gelassen werden, weil mit dem Dart-Websocket-Server einige der grundlegenden Anforderungen nicht umzusetzen waren. Zum einen unterstützt Dart keine JSON-over-TCP -Kommunikation, wie sie für die Abfrage des JSON-Datenstroms vom Rohdatenserver erforderlich ist. Und zum anderen gab es zum Zeitpunkt der Implementierung noch keinen Redis-Client für Dart. Der publish/subscribe Mechanismus der Redis-Datenbank wird für die Verteilung der Positionsupdates benötigt.\\

Deshalb wird nur die Client-Anwendung in Dart implementiert (in Abschnitt \ref{HTML5-Client in Dart}), während als Server der Javascript-Websocket-Server genutzt werden soll. Das ist allerdings nicht ohne Anpassung möglich, weil der socket.io-Websocket-Server auf dem Client die socket.io.js-Datei erfordert, die in Dart nicht zur Verfügung steht. \\Deshalb wird neben dem socket.io-Server ein zweiter Server implementiert, der eine Websocket-Verbindung nach der HTML5-Websocket-API-Spezifikation aufbaut (in Abschnitt \ref{HTML5-Server}), die in Dart clientseitig mit dem Paket dart:html unterstützt wird. Die Änderung der Server-Anwendung von einem socket.io-Socket-Server zu einem HTML5-Websocket-Server ist in node.js mit dem websocket-Modul  \footnote{https://github.com/Worlize/WebSocket-Node/wiki/Documentation} recht einfach möglich.
\section{Das Problem der Vergleichbarkeit}\label{Vergleichbarkeit}
An dieser Stelle stellt sich die Frage, ob beide Serverlösungen vergleichbare Ergebnisse liefern. Denn Unterschiede zwischen den node.js-Servern (socket.io vs. HTML5) würden in das Ergebnis des Vergleichs zwischen den Clients (Dart vs. Javascript) mit einfliessen. Deshalb wird für den HTML5-kompatiblen Websocket-Server noch ein Javascript-Client geschrieben, der dann direkt vergleichbar ist mit dem in Dart geschriebenen Client.
Zunächst ist zu beurteilen, inwieweit die Javascript-Implementierung durch den Verzicht auf das socket.io-Framework, das in der ersten Lösung verwendet wird, ausgebremst wird. Das geschieht über einen Performance-Vergleich in Abschnitt \ref {socket.io- vs html5-Server}.
Es werden also zwei Vergleiche durchgeführt (siehe auch Abbildung \ref{tab:uebersicht}) 
\begin{itemize}
\item In Javascript wird der socket.io-Websocket gegen den HTML5-Websocket getestet.
\item Unter Verwendung des HTML5-Websockets wird der Javascript-Client gegen den Dart-Client getestet
\end{itemize}
Tabelle \ref{tab:uebersicht} zeigt, wie die Lösung in node.js mit socket.io nicht unmittelbar sondern mittelbar über die Javascript-Lösung mit HTML5-Websocket gegen die Implementierung mit einem Google Dart Client getestet wird.
\begin{figure}[H]
  \centering
  \includegraphics[width=6in]{images/tabelleblaurot.png}
  \caption[Übersicht über ausgeführte Server-und Clientimplementierungen]{Übersicht über ausgeführte Server-und Clientimplementierungen}
  \label{tab:uebersicht}
\end{figure}

%\renewcommand{\arraystretch}{2}

%\begin{table}[!hbt]\vspace{1ex}\centering
%\begin{tabular}{| l| m{3.5cm}||c|c|}\cline{3-4}

%\multicolumn{2}{c||}{}&\multicolumn{2}{c|}{HTTP-Client}\\\cline{3-4}
%\multicolumn{2}{c||}{}& Javascript& Google Dart\\\hline\hline
%\multirow{2}*{\rotatebox{90}{HTTP-Server}}& socket.io-Websocket-Server  (\ref{socket.io-Server}) &  socket.io-Client  (\ref{socket.io-Client})& \includegraphics[width=0.2in]{images/x_red.jpeg}\\\cline{2-4}
%&HTML5-Websocket-Server (\ref{HTML5-Server}) & HTML5-Client (\ref{HTML5-Client in Javascript}) & HTML5-Client  (\ref{HTML5-Client in Dart})\\\hline
%\multicolumn{4}{c}{}\\
%\end{tabular}
%\caption[Übersicht über Server-und Clientimplementierungen in dieser Arbeit]
%{Übersicht über Server-und Clientimplementierungen in dieser Arbeit\\}
%\vspace{2ex}
%\label{tab:uebersicht}
%\end{table}
\newpage
%-------------------------------------------------------------------------------------------------------------------------------
\section{Beschreibung der ausgeführten Implementierungen}
In der ersten Implementierung werden Lösungen entwickelt für die in den Anforderungen beschriebenen Aufgaben. In allen weiteren Implementierungen werden diese Lösungen möglichst übernommen und andernfalls eine Alternative entwickelt.
%-------------------------------------------------------------------------------------------------------------------------------
\subsection{socket.io-Server}\label{socket.io-Server}
Die zu entwickelnde Serveranwendung hat grundsätzlich zwei Aufgaben: 
\begin{enumerate}
\item Daten über eine JSON-over-TCP-Verbindung vom Rohdatenserver abzurufen und
\item einen Websocket zu betreiben, der die Daten an Websocket-Clients weitergibt
\end{enumerate}
Weil node.js singlethreaded ist (vgl. \ref{Node.js}), würde der Server beide Aufgaben in einem einzigen Prozess bearbeiten. Um das Potential an Parallelverabeitung eines Dualcore oder Multicore-Servers zu nutzen, ist des daher sinnvoll, für jede Aufgabe einen eigenen Prozess zu generieren. Dazu wurde das node.js-Modul child\_process genutzt. Die Start-Datei master.js generiert damit zuerst einen Prozess, der den AIS-Daten-Client (aisData-client.js) abzweigt, um Daten vom Rohdaten-Server abzufragen und anschließend einen worker-Prozess (worker.js), um einen Websocket -Server für Client-Verbindungen zur Verfügung zu stellen (siehe Listing \ref{master.js}).
\begin{lstlisting}[caption=Generierung von Kindprozessen in master.js, firstnumber=16, label=master.js]
/* AIS-Client - Process*/
  child.fork(path.join(__dirname, 'aisData-client.js'));

/*worker- Process*/
  child.fork(path.join(__dirname, 'worker.js'));
\end{lstlisting}
Bei der Weitergabe der Daten durch den worker-Prozess sind zwei Fälle zu unterscheiden:
\begin{itemize}
\item ein Client verbindet sich neu oder ändert den Kartenausschnitt. In diesem Fall (Vessels-in-Bounds-Request) sind die Schiffs-und Positionsdaten aller sich aktuell in diesem Bereich (Bounds) befindlichen Schiffe an den Client zu senden (Kapitel \ref{Vessels-in-Bounds-Request}).
\item ein Schiff sendet ein Positions-Update, das an alle Clients verteilt werden muss, deren Kartenausschnitt die betreffende Schiffsposition enthält. Dieses Ereignis wird im Folgenden Vessel-Position-Update genannt (Kapitel \ref{Vessel-Position-Update}).
\end{itemize}

\subsubsection{Vessels-in-Bounds-Request}\label{Vessels-in-Bounds-Request}
Der Vessels-in-Bounds-Request macht eine Zwischenspeicherung der Daten unumgänglich. Denn es sollen für jeden Bereich weltweit sofort alle Schiffe zurückgegeben werden können, die in den vergangenen 10 Minuten ihre Position aus diesem Bereich gesendet haben. Wegen der großen Anzahl gleichzeitig empfangener Schiffe (weltweit ca. 60.000) und der Notwendigkeit, einen geographischen Index zu verwenden, wird einer persistenten gegenüber einer transienten Speicherung der Vorzug gegeben. 
\\Für die Persistierung wird hier MongoDB verwendet, weil MongoDB als NoSQL-Datenbank mit geringem Overhead schnelle Antwortzeiten und außerdem einen geographischen Index bietet. Der Serverprozess in aisData-client.js schreibt die Daten (siehe listing \ref{write in Mongo}) in eine MongoDB-Collection namens ‘vesselsCollection’. Die MMSI eines Schiffes ist eindeutig und wird als unique key verwendet (siehe Abschnitt \ref{Statische Schiffsdaten}). Über die Option upsert:true wird der Mongo Datenbank mitgeteilt, dass entweder ein insert-Befehl oder, falls die mmsi bereits in der MongoDB-Collection vorhanden ist, ein update-Befehl auf den entsprechenden Datensatz auszuführen ist. 
\begin{lstlisting}[caption=Schreiben in die Datenbank in aisData-client.js, label=write in Mongo]
vesselsCollection.update(
  { mmsi: obj.mmsi },
  { $set: obj },
  { safe: false, upsert: true }
  );
\end{lstlisting}
In Listing \ref{2d-Index} ist zu sehen, wie über das Schlüsselwort “2d” der MongoDB-Geo-Index auf das Feld ‘pos’ mit den Koordinaten des Schiffes gesetzt wird. Damit ist garantiert, dass nicht jede Anfrage des Servers an die Datenbank sämtliche Datensätze durchlaufen muss, um die Schiffe in einem bestimmten geographischen Ausschnitt zu finden. Aufbau und Unterhalt des Geo-Indexes findet im aisData-client-Prozess statt, der schreibend auf die Datenbank zugreift.
\begin{lstlisting}[caption=Aufbau des Geo-Indexes in aisData-client.js, label= 2d-Index]
  vesselsCollection.ensureIndex({ pos: "2d", sog: 1, time_received: 1 }, function(err, result) {... });
  \end{lstlisting}
Dabei handelt es sich um einen zusammengesetzten Index, weil neben der Geo-Position auch die Geschwindigkeit und der Zeitpunkt des Empfangs der Nachricht Filterkriterien sind, wenn der zweite Prozess (worker.js) lesend auf die Datenbank zugreift. In Listing \ref{lst:query Mongo} ist zu sehen, wie der Prozess worker.js mit den vom Client in einem Vessels-in-Bounds-Request übermittelten Geo-Daten die MongoDb anfragt.
  \begin{lstlisting} [caption=Vessel-in-Bounds-query in worker.js, label=lst:query Mongo]
  var vesselCursor = vesselsCollection.find({
    pos: { $within: { $box: [ [bounds._southWest.lng,bounds._southWest.lat], [bounds._northEast.lng,bounds._northEast.lat] ] } },
    time_received: { $gt: (new Date() - 10 * 60 * 1000) },
    sog: { $exists:true },
    sog: { $gt: zoomSpeedArray[zoom]},
    sog: {$ne: 102.3}
  });
  vesselCursor.toArray(function(err, vesselData)  {
   client.sendUTF(JSON.stringify({ type: 'vesselsInBoundsEvent', vessels: vesselData}));
});
\end{lstlisting}

\subsubsection{Vessel-Position-Update}\label{Vessel-Position-Update}
Für die Kommunikation eines Vessel-Position-Updates (AIS-Nachrichtentyp 1-3) zwischen dem aisData-client.js-Prozess und dem worker.js-Prozess wird der publish/subscribe-Mechnismus einer Redis-Datenbank genutzt\footnote{http://redis.io/topics/pubsub}. Der aisData-client.js-Prozess publiziert jedes Positions-Update auf dem Kanal ‘vessel-Pos’ der Redis-Datenbank. Der worker.js-Prozess meldet sich als subscriber am Kanal ‘vessel-Pos’ der Redis-Datenbank an und wird so bei jedem Positions-Update benachrichtigt.
Um diese Nachricht an die betroffenen Websocket-Clients weiterzuleiten, ist eine serverseitige Verwaltung der Clients notwendig. Die Serveranwendung muss bei jeder Positionsmeldung wissen, welche Clients benachrichtigt werden müssen. Die Client-Verwaltung ist ein Feature des socket.io-Paketes. Für jeden Client wird bei der Registrierung zusätzlich das Zoomlevel der Karte und die Koordinaten gespeichert.
\begin{lstlisting}[caption= Speichern der übermittelten Client-Daten in worker.js, label=Speichern der übermittelten Client-Daten in worker.js]
io.sockets.on('connection', function(client) {
      log(' Connection from client accepted.');
      client.on('register', function(bounds, zoom) {
          client.set('zoom', zoom);
          client.set('bounds', bounds, function() {
              getVesselsInBounds(client, bounds, zoom);
          });
      });
      client.on('unregister', function() {
          client.del('bounds');
          client.del('zoom');
      });
});
  \end{lstlisting}
  Bei jedem Vessel-Position-Update, das der worker.js-Prozess empfängt, geht er die Liste der Clients durch und benachrichtigt diejenigen, in deren Bereich das Positions-Update fällt.
\begin{lstlisting}[caption= Weiterleitung von Positions-Updates an Websocket-Clients in worker.js, label= Weiterleitung von Positions-Updates an Websocket-Clients in worker.js]
 redisClient.on('message', function(channel, message) {
    if (channel == 'vesselPos')  {
      ...
      var json = JSON.parse(message);
      ...
      var clients = io.sockets.clients();
      var lon = json.pos[0];
      var lat = json.pos[1];
      var sog = json.sog/10;
      var cog = json.cog/10;
      clients.forEach(function(client) {
        client.get('bounds', function(err, bounds) {
          if (bounds != null && lon != null && lat != null) 
          {
            /* check, if Client-Connection is affected by Vessel-Position-Update */
            if (positionInBounds(lon, lat, bounds)) 
            {
              client.get('zoom', function(err, zoom) 
              {
                if(sog !=null && sog > (zoomSpeedArray[zoom]) && sog != 102.3)
                {
                  client.emit('vesselPosEvent', message);
                }
          ...
  });
\end{lstlisting}
%-------------------------------------------------------------------------------------------------------------------------------------------
\begin{figure}[H]
  \centering
  \includegraphics[width=6in]{images/Javascript-Dateien-undObjekte.png}
  \caption[Übersicht Javascript-Files]{Schematische Übersicht der implementierten Javascript-Dateien und -Objekte. In der Darstellung wird die Notation des UML-Klassendiagramms verwendet, was natürlich (Klassen existieren nicht in Javascript) nicht als solches gelesen werden kann. Weil aber die Anwendung obektorientiert konzipiert ist, war dieser Ansatz sowohl während der Entwicklung als auch bei der Beschreibung der Anwendung sehr nützlich. Die Attribute und Methoden sind nicht vollständig dargestellt, sondern so ausgewählt, dass Zusammenhänge und Abhängigkeiten erkennbar werden.
Dasselbe gilt für Abbildung  \ref{fig:Übersicht Dart-Files}, die den Dart-Client schematisch darstellt. }
  \label{fig:Übersicht Javascript-Files}
\end{figure}
%-------------------------------------------------------------------------------------------------------------------------------
\subsection{socket.io-Client}\label{socket.io-Client}
Der socket.io-Client hat folgende Aufgaben.
\begin{enumerate}
\item \label{itm:first} Es ist eine html-Seite zu erstellen, die die benötigten Source-Dateien lädt und eine HTML-Struktur aufbaut mit den DOM-Elementen für die Karte.
\item  \label{itm:second}URL-Parameter sollen optional übergeben werden können.
\item  \label{itm:third}Optionen sollen zentral an einer Stelle der Anwendung geändert werden können.
\item  \label{itm:fourth} Eine Karte auf Basis des auf dem unternehmenseigenen Server gehosteten Kartenmaterials mit Navigations- und Infobereichen ist in den Kartenbereich zu rendern.
\item  \label{itm:fifth} Zum Socket.io-Websocket-Server soll eine Verbindung aufgebaut werden, die
\begin{enumerate}
\item  \label{itm:fifthA}Vessel-In-Bounds- und Vessel-Position-Events empfängt und
\item \label{itm:fifthB}bei Positionsänderungen auf der Karte eine register-Nachricht sendet
\end{enumerate}
\item \label{itm:sixth} Aus den empfangenen JSON-Daten sind geeignete Objekte zu erstellen und zu speichern.
\item \label{itm:seventh}Die Objekte sind als Features auf die Karte zu rendern
\item \label{itm:eighth} Objekte, die den Status ‘Moving’ haben, sind entsprechend ihrer Geschwindigkeit zu animieren.
\end{enumerate}
Punkt \ref{itm:first} geschieht in der Datei ais-socket.io.html, die zum Start der Anwendung vom Browser-Client aufgerufen wird. Dort werden die benötigten Javascript- und css-Dateien eingebunden, inklusive der JQuery- und der Leaflet-Bibliothek. Nach dem Laden führt der Browser die Anweisungen innerhalb der Funktion \$(document).ready(function() {...} in der Datei ais-socket.io.js aus. Falls Url-Parameter übergeben worden sind für initialen Zoomlevel und Kartenausschnitt (Punkt \ref{itm:second}) werden sie nun mit der Funktion getParam(name) aufgegriffen, sonst wird ein Defaultwert für zoom und bounds benutzt. Wie unter Punkt \ref{itm:third} gefordert, kann dieser Defaultwert und alle weiteren Einstellungen (z.B. Server-IP, Server-Port, Map-Server-Url (Punkt \ref{itm:fourth})) in dieser Datei zentral angepasst werden.
Als Javascript-Bibliothek zur Darstellung der Schiffe auf der Karte wurde die Leaflet-Bibliothek (siehe Abschnitt \ref{Leaflet}) ausgewählt. Die Cockpit-Anwendung der Vesseltracker-GmbH nutzt die älteren OpenLayers-Bibliotheken, diese sind jedoch im Vergleich sehr viel sperriger in der Nutzung und werden inzwischen weniger aktiv von der Community weiterentwickelt und verbessert.
Mithilfe der Leaflet-Bibliothek wird die Karte als Javascript-Objekt in der Datei LeafletMap.js realisiert. Dazu wird das ‘Revealing Module Pattern’ genutzt, mit dem sich die API der Karte von ihrer internen Implementierung trennen lässt. Nur die in der return-Klausel zurückgegebenen Funktionen bilden die öffentliche Schnittstelle des Karten-Objektes (Punkt \ref{itm:fourth}).

\begin{lstlisting}[caption= ‘Revealing Module Pattern’ in LeafletMap.js, label=LeafletMap.js]
var LMap = function(){
  var map, featureLayer, tileLayer, zoom, socket, boundsTimeout, boundsTimeoutTimer;
  function init(elementid, initOptions, mapOptions, tileLayerOptions) { ... }
  function changeRegistration() { ... } 
  function getMap(){ ... }
  function getZoom(){ ... }
  function addToMap(feature, animation, popupContent){ ... } 
  function removeFeatures(vessel){ ... }
  return {
    init: init,
    getMap: getMap,
    getZoom: getZoom,
    addToMap: addToMap,
    removeFeatures: removeFeatures }
}();
\end{lstlisting}
Nach dem Initialisieren der Karte wird die Websocket-Verbindung (Punkt \ref{itm:fifth}) hergestellt (siehe Listing \ref{ais-socket.io.js}). Für den Empfang der Nachrichten des Websocket-Servers (Punkt \ref{itm:fifthA}) genügen dazu zwei Listener: 
\begin{lstlisting}[caption=Client-Seite der socket.io-Websocket-Verbindung in ais-socket.io.js,  label=ais-socket.io.js]
 var socket = io.connect('http://'+WEBSOCKET_SERVER_LOCATION+':'+WEBSOCKET_SERVER_PORT);
socket.on('vesselsInBoundsEvent', function (data) {...}
socket.on('vesselPosEvent', function (data) {...}
\end{lstlisting}

Um eine Liste aller im Kartenbereich befindlichen Schiffe vom Server zu bekommen, muss der Client eine ‘register’-Nachricht mit den aktuellen Bounds an den Server senden (Punkt \ref{itm:fifthB}).
Dies geschieht einmal nach dem Intialisieren der Karte und soll anschließend durch den von der Leaflet-Map nach jeder Änderung des Kartenausschnitts getriggerten moveend-Event ausglöst werden, bzw. spätestens nach der über BOUNDS\_TIMEOUT konfigurierten Zeitspanne. Weil dieser Event innerhalb des LeafletMap-Objektes auftritt, wird dem LeafletMap-Objekt bei der Initialisierung eine Referenz auf die Websocket-Connection übergeben. Diese Lösung ist unproblematisch, weil beim Verlust der socket-Verbindung ohnehin ein Reload der Seite stattfindet.\\
Um geeignete Objekte aus den Ais-Messages zu erstellen (Punkt \ref{itm:sixth}), wird in der Datei Vessel.js eine Konstruktor-Funktion zur Verfügung gestellt, mit der Instanzen von Vessel-Objekten generiert werden können. Diese Instanzen werden in einem assoziativen Array, also einem Objekt, namens ‘vessels’ in ais-socket.io.js gespeichert. Beim Empfang eines Vessel-Position-Events kann mit vessels[mmsi] nach dem passenden vessel-Objekt zum Update gesucht werden.
Schließlich sind die Vessel-Objekte auf die Karte zu rendern (Punkt \ref{itm:seventh}). Dazu wird in Vessel.js eine asynchrone Funktion genutzt (siehe Listing \ref{vessel.paintToMap}), mit der zuerst je nach Status (moving / not moving) und Zoomlevel unterschiedliche Features (Polygon, Triangle, Speedvector, Circle) für ein Schiff erstellt und auf die Karte gerendert werden. Anschließend wird das Vessel-Objekt mit seinen Features im assoziativen Array, bzw. Objekt ‘vessels’ gespeichert. Das Speichern ist notwendig, um die Features bei Vessel-Position-Updates von der Karte zu entfernen, bevor sie an eine neue Position gerendert werden.
\begin{lstlisting}[caption=Aufruf der public function paintToMap des Vessel-Objekts in ais-socket.io.js, label=vessel.paintToMap]
          vessel.paintToMap(LMap.getZoom(), function(){
              vessels[vessel.mmsi] = vessel;
          });
\end{lstlisting}

Für díe letzte Aufgabe, die Animation (Punkt \ref{itm:eighth}), wird das Polygon-Objekt des Leaflet-Frameworks erweitert zur Klasse L.AnimatedPolygon. Ein Polygon in Leaflet ist eine Polyline, die mehrere Punkte auf der Karte verbindet. Die Animation eines Punktes entlang einer Linie mit einer bestimmten Geschwindigkeit wurde aus dem Leaflet-Plugin L.AnimatedMarker \footnote{https://github.com/openplans/Leaflet.AnimatedMarker} übernommen. Die dort präferierte css3-Transition zur Animation konnte aber nicht verwendet werden, weil dazu ein Objekt im DOM-Baum identifiziert werden muss. Das von Leaflet für ein Polygon erstellte DOM-Objekt (svn-Graphik) ist aber durch die Leaflet-Bibliothek gekapselt. Der Versuch, die jeweilige svn-Grafik im DOM-Baum für die Animation zu selektieren, ist gescheitert. \\
Deshalb musste auf eine Lösung zurückgegriffen werden, die die MapFeatures des Schiffes (Polygon und Richtungsdreieck) in jedem Animationsschritt neu berechnet und rendert.

Ein Schiffspolygon berechnet sich aus der Schiffsposition (Positionsangabe in der AIS-Nachricht) und aus der relativen Position des AIS-Transceivers an Bord (Abstand zum Bug, zum Heck, nach Backbord und nach Steuerbord). Um zu wissen, in welche Richtung der Bug eines Schiffes zeigt, wird die AIS-Angabe zur Fahrtrichtung  bei fahrenden Schiffen verwendet (cog = Course over Ground). Aus dieser Richtungsangabe und der übermittelten Geschwindigkeit (sog = Speed over Ground) wird ein Speedvector berechnet, der als Linie auf die Karte gezeichnet wird. Er ist genauso lang wie die Strecke, die das Schiff bei kontinuierlicher Weiterfahrt in den nächsten 30 Sekunden zurücklegen wird. Bei der Erstellung des AnimatedPolygon-Objektes wird dieser Speedvector übergeben, damit das Polygon entlang dieser Linie verschoben werden kann. Aus den Optionen distance und interval wird bestimmt, in wieviele Teilschritte der Vektor unterteilt wird und wieviel Zeit zwischen zwei Animationsschritten liegt. Nach jedem Animationsschritt wird das Polygon neu berechnet, von der Karte gelöscht und neu gezeichnet.

%-------------------------------------------------------------------------------------------------------------------------------------------

\subsection{HTML5-Server}\label{HTML5-Server}
Diese Server-Implementierung soll genau dieselbe Funktionalität besitzen wie die socket.io-Server-Implementierung (\ref{socket.io-Server}). Lediglich der Websocket socket.io wird durch einen node.js-Websocket nach der HTML5-Websocket-Spezifikation ausgetauscht\footnote{https://github.com/Worlize/WebSocket-Node} . Dazu wird in der Datei worker.js das entsprechende Paket (‘websocket’) eingebunden.
Einige Features des socket.io-Paketes müssen jetzt selbst organisiert werden: 
\begin{itemize}
\item die Clientverwaltung erfolgt in einem Array ‘clients’, in dem zu jeder Zeit alle verbundenen Websocket-Clients mitsamt ihren Attributen stehen. 
\item in der Websocket-API können keine eigenen Events definiert werden (siehe API-Dokumentation\footnote{https://github.com/Worlize/WebSocket-Node/wiki/Documentation}). Deshalb wird der message-Event genutzt und der Name der aufzurufende Funktion wird innerhalb der message als “type” übermittelt. Unten ist eine identische Nachricht in den beiden unterschiedlichen Formaten zu sehen.
\end{itemize}

\begin{lstlisting}[caption= vom socket.io-Server gesendete message, label=socket.io-message]
[{"_id":"50d9fdb4bcc2e678a9278c18","aisclient_id":57,"callsign":"OVYC2  ","cog":285,"dest":"HAMBURG             ","dim_bow":70,"dim_port":10,"dim_starboard":4,"dim_stern":30,"draught":54,"imo":"9363170","mmsi":220515000,"msgid":1,"name":"RIKKE THERESA       ","nav_status":0,"pos":[9.8375,53.54885],"rot":0,"ship_type":80,"sog":10.3,"time_captured":1366734056000,"time_received":1366733855248,"true_heading":286}]
\end{lstlisting}

\begin{lstlisting}[caption= vom Websocket-Server gesendete message, label=websocket-message]
message origin=ws://127.0.0.1:8090, data={"type":"vesselsInBoundsEvent","vessels":[{"_id":"50d9fdb4bcc2e678a9278c18","aisclient_id":57,"callsign":"OVYC2 ","cog":285,"dest":"HAMBURG ","dim_bow":70,"dim_port":10,"dim_starboard":4,"dim_stern":30,"draught":54,"imo":"9363170","mmsi":220515000,"msgid":1,"name":"RIKKE THERESA ","nav_status":0,"pos":[9.8375,53.54885],"rot":0,"ship_type":80,"sog":10.3,"time_captured":1366733896000,"time_received":1366733855248,"true_heading":286}]}

\end{lstlisting}

%-------------------------------------------------------------------------------------------------------------------------------------------
\subsection{HTML5-Client in Javascript}\label{HTML5-Client in Javascript}
Die HTML5-Client-Implementierung bietet die gleiche Funktionalität wie die socket.io-Client-Implementierung und unterscheidet sich nur marginal.
 \begin{itemize}
 \item wie in Listing \ref{websocket-message} zu sehen, muss der HTML5-Client zuerst den message-type abfragen, um die Daten korrekt zuzuordnen.
 \item weil dem Leaflet-Map-Objekt die Websocket-Connection als Parameter übergeben wird, muss der Aufbau der Websocket-Connection abgewartet werden. Deswegen wird die Map-Initialisierung erst durch den onopen-Event des Websockets angestossen.
\end{itemize}
%-------------------------------------------------------------------------------------------------------------------------------------------


\begin{figure}[H]
  \centering
  \includegraphics[width=6.2in]{images/Dart-DateienUndKlassen.png}
  \caption[Übersicht Dart-Files]{Schematische Übersicht der Dart-Dateien und -Klassen, sowie der verwendeten Javascript-Dateien, siehe auch Anmerkungen zu Abbildung \ref{fig:Übersicht Javascript-Files}}
  \label{fig:Übersicht Dart-Files}
\end{figure}


\subsection{HTML5-Client in Google Dart}\label{HTML5-Client in Dart}
Die Implementierung des HTML5-Websocket-Clients in Dart orientiert sich an der Implementierung des HTML5-Websocket-Clients in Javascript. Oberstes Ziel dabei ist es, eine mindestens gleichwertige Funktionalität zu erreichen unter Ausnutzung der sprachspezifischen Vorteile von Google Dart. \\

Die Modularisierung, also die Verteilung der Objekte und Funktionalitäten auf mehrere Dateien, ist in der Dart-Client-Implementierung ähnlich gestaltet wie in Javascript (siehe Abbildungen \ref{fig:Übersicht Javascript-Files} und \ref{fig:Übersicht Dart-Files}). Die verwendeten Javascript-Dateien ‘leaflet-src.js’, ‘L.AnimatedPolygon.js’ und ‘L.Control.Mouseposition.js’ sind identisch mit denen des Javascript-Clients (Abschnitt \ref{HTML5-Client in Javascript}). Um sie aus Dart heraus nutzen zu können, wird das Dart-Paket js-interop (siehe Abschnitt \ref{js-interop}) eingebunden.
Eine Dart-Anwendung wird über die main-Funktion (in ais-htlml5.dart) gestartet. Die main-Funktion initialisiert den HTML5-Websocket-Client. Im Falle eines erfolgreichen Verbindungsaufbaus mit dem HTML5-Websocket-Server wird die Callback-Funktion ausgeführt, die das LeafletMap-Objekt erstellt. Beide Objekte sind Singletons und bleiben über die Laufzeit der Anwendung erhalten.
Das LeafletMap-Objekt kapselt für die Anwendung den Zugriff auf das Javascript L.Map-Objekt der leaflet.js-Bibliothek (siehe Listing \ref{LeafletMapConstructor}). Dafür wechselt es in den Javascript-Kontext der Anwendung und initialisiert in diesem ein L.Map-Objekt und ein L.LayerGroup-Objekt. Beide Objekte werden mit der Anweisung js.retain(...) im Javascript-Kontext als globale Variable eingeführt, so dass sie für jede folgende Funktion im Javascript-Kontext zur Verfügung stehen.
In der Gegenrichtung muss eine Möglichkeit existieren, vom Javascript-Kontext der Anwendung auf den Dart-Code zuzugreifen. Ein Beispiel dafür ist der ‘moveend’-Listener, der Teil der leaflet.js-Bibliothek ist. Für den Javascript-Listener wird ein Dart-Callback.many-Objekt erstellt (moveendHandler(e)), das die changeRegistration-Funktion im Dart-Kontext ausführt. Das Callback.many-Objekt kann im Unterschied zum Callback.once-Objekt mehrmals aufgerufen werden. Das heißt, jedes Mal, wenn im Javascript-Kontext der  ‘moveend’-Listener einen  ‘moveend’-Event registriert, ruft er über den Handler die Dart-Funktion changeRegistration() auf. Diese Funktion sendet an den HTML5-Websocket eine Nachricht vom Typ ‘register’ mit den aktuellen Bounds der Karte.

\begin{lstlisting}[caption=Konstruktor des LeafletMap-Objektes mit Zugriff auf den Javascript-Kontext, label=LeafletMapConstructor]
LeafletMap(String elementid, js.Proxy mapOptions, js.Proxy initOptions, js.Proxy tileLayerOptions){
    boundsTimeout = initOptions['boundsTimeout']*1000;
    js.scoped(() {
        map = new js.Proxy(js.context.L.Map, elementid, mapOptions);
        featureLayerGroup = new js.Proxy(js.context.L.LayerGroup);
        map.addLayer(featureLayerGroup);
        js.retain(featureLayerGroup);
        var tileLayer = new js.Proxy(js.context.L.TileLayer,tileLayerOptions['tileURL'], tileLayerOptions);
        map.addLayer(tileLayer);
        var mouseOptions = initOptions['mousePosition'];
        if( mouseOptions != false)
        {
          var mousePosition = new js.Proxy(js.context.L.Control.MousePosition, mouseOptions);
          mousePosition.addTo(map);
        }
        map.setView(new js.Proxy(js.context.L.LatLng, initOptions['lat'], initOptions['lon']),initOptions['zoom']);
        js.retain(map);
        map.on('moveend', new js.Callback.many(moveendHandler));
      });
     changeRegistration();
  }
\end{lstlisting}
Der Websocket-Server antwortet auf diese Nachricht mit einer Nachricht vom Typ “VesselsInBoundsEvent” (siehe Listing \ref{websocket-message}).\\
Diese Nachricht und die “VesselPositionEvent”-Nachricht empfängt und verarbeitet der Dart-Client genauso wie der Javascript-Client in Listing \ref{vessel.paintToMap}. Bei der Ausführung der paintToMap-Methode des Vessel-Objektes allerdings besteht ein wichtiger Unterschied darin, dass in Javascript direkt Map-Feature-Objekte der Leaflet.js Bibliothek (L.Polyline, L.AnimatedPolygon und L.CircleMarker) erzeugt werden, wohingegen in Dart der Konstruktor einer Unterklasse (Polyline, AnimatedPolygon oder CircleMarker) von MapFeature aufgerufen wird. Die Konstruktoren aller Unterklassen von MapFeature wechseln in den Javascript-Kontext, erstellen dort über einen Proxy ein entsprechendes Objekt aus der Leaflet-Bibliothek und stellen dieses als Attribut des MapFeature-Objektes im Dart-Kontext zur Verfügung (siehe Beispiel Klasse Polyline in Listing \ref{Constructor Polyline}).

\begin{lstlisting}[caption= Konstructor des Dart-Objekts Polyline, label = Constructor Polyline]
class Polyline extends MapFeature{
  Polyline(List<Coord> vectorPoints, Map options) {
    js.scoped(() {
      var latlng = js.context.L.LatLng;
      var points =js.array([]);
      for(var x = 0;x < vectorPoints.length; x++)
      {
        var lat =  vectorPoints[x].latitude;
        var lng = vectorPoints[x].longitude;
        points.push(new js.Proxy(latlng,lat,lng ));
      }
      var lineOptions = js.map(options);
      _mapFeature= new js.Proxy(js.context.L.Polyline, points, lineOptions);
      js.retain(_mapFeature);
    });
  }
}
\end{lstlisting}
Auf diese Weise erhält jedes Vessel-Objekt im Array ‘Vessels’, der alle auf der Karte dargestellten Schiffe enthält, als Attribut .feature und/oder .polygon eine Assoziation zu einer Dart-Klasse, die über einen js.Proxy die Verwaltung des dazugehörigen Javascript-Objektes im DOM-Baum für die Anwendung kapselt.

Nach demselben Prinzip funktionieren Popups in der Anwendung.

Um Popups auf den MapFeature-Objekten über MouseEvents öffnen und schließen zu können, werden auf den mouseover und den mouseout-EventListener des Javascript-Proxies eines jeden MapFeature-Objektes EventHandler als Dart-Callback.many-Funktionen registriert (siehe Listing \ref{EventHandling}). Diese Callback.many-Funktionen führen im Dart-Kontext alle notwendigen Anweisungen aus. Zum Beispiel erstellt der MouseoverHandler ein Dart-Popup-Objekt, das im Konstruktor einen Proxy zu einem L.Popup-Objekt erstellt und als Instanzvariable speichert. Operationen auf dem L.Map-Objekt müssen über das Dart LeafletMap-Objekt ausgeführt werden, das den Javascript-Proxy zum L.Map-Objekt verwaltet. Im Falle des onMouseoverHandlers, wird das Dart-Popup-Objekt der Funktion LMap.openPopup(popup) übergeben, die Funktion wechselt in den Javascript-Kontext und öffnet auf dem L.Map-Objekt den Javascript-Proxy des Popup-Objektes (siehe Listing \ref{LeafletMap-Objekt})
\\
\begin{lstlisting}[caption=EventHandling mithilfe von Callback-Funktionen, label= EventHandling]
void addListeners(){
    onMouseoutHandler(e){
      LMap.closePopup();
    }
    onMouseoverHandler(e){
      var ll = e.latlng;
      ll =  new js.Proxy(js.context.L.LatLng ,ll.lat, ll.lng);
      var popupOptions = {'closeButton': false,
                        'autoPan': false,
                        'maxWidth': 150, 
                        'offset' : [50,-50]};
      var popup = new Popup(ll, popupContent, popupOptions);
      LMap.closePopup();
      LMap.openPopup(popup);
    }
    
    _callbacks.add(new js.Callback.many(onMouseoverHandler));
    _callbacks.add(new js.Callback.many(onMouseoutHandler));
    
    _mapFeature.on('mouseover', _callbacks[0]);
    _mapFeature.on('mouseout', _callbacks[1]);
  }
  \end{lstlisting}
  \begin{lstlisting}[caption=Zugriff auf das L.Map-Objekt (im Javascript-Kontext) über das LeafletMap-Objekt (im Dart-Kontext) in LeafletMap.dart, label= LeafletMap-Objekt]
class LeafletMap {
  js.Proxy map;
  ...
  closePopup(){
    js.scoped((){
      map.closePopup();
    });
  }

  openPopup(Popup popup){
    js.scoped((){
      map.openPopup(popup._popup);
    });
  }
...
}  
  \end{lstlisting}

%--------------------------------------------------------------------------------------------------------------------------------------------

\chapter{Ergebnisse}

\section{Evaluation der Anwendung}\label{Evaluation der Anwendung}
Alle funktionalen Anforderungen sind im Prototyp der Anwendung in Javascript mit dem node.js-Framework und dem socket.io-Websocket (Abschnitt \ref{socket.io-Server} und \ref{socket.io-Client}) vollständig umgesetzt. Im Screenshot des Hamburger Hafens (Abb. \ref{Hafen Hamburg}) kann man erkennen, dass liegende Schiffe als Kreise dargestellt werden und fahrende Schiffe als Richtungsdreiecke. Wurden Masse übermittelt (AIS-Nachricht vom Typ 4), sind maßstabsgetreue Polygone eingezeichnet, deren Farbe den Schiffstyp kennzeichnet.
Ein Popup mit Detailinformationen zum dem Schiff links daneben ist geöffnet. In dieser Zoomstufe ist die Animation bereits aktiv, das heißt für den Betrachter, dass alle angezeigten Richtungsdreiecke ununterbrochen in Bewegung sind.

\begin {figure}[H]
\begin{center}
  \includegraphics[width=6in]{images/Hamburg.png}
\end{center}
\caption{Anzeige aller Schiffe im Hamburger Hafen}
\label{Hafen Hamburg}
\end {figure}
Das zweite Beispiel zeigt die Situation bei niedriger Zoomstufe (Abb. \ref{Nordsee}). Oben links in der Karte ist ein Hinweis eingeblendet, dass nur Schiffe mit einer Geschwindigkeit über 12 Knoten angezeigt werden. Die Verteilung des Schiffsverkehrs lässt sich gut erkennen, während die große Zahl hafenliegender Schiffe ausgespart bleibt, um die Performance des Browser-Clients zu erhalten. Aus demselben Grund ist die Animation ausgeschaltet. Durch die große Anzahl empfangener Positionsmeldungen, ist die Darstellung trotzdem bewegt.

\begin {figure}[H]
\begin{center}
  \includegraphics[width=6in]{images/zoomout.png}
\end{center}
\caption{Auf schnell fahrende Schiffe reduzierte Anzeige am Beispiel der Nordsee}
\label{Nordsee}
\end {figure}

\begin {wrapfigure}[9]{r}{3in}
\begin{center}
  \includegraphics[width=2in]{images/Schleppen.png}
 \caption{Schleppmanöver}
  \end{center}
 \label{Schleppmanöver}

\end {wrapfigure}


Schließlich ist in hohen Zoomstufen eine Beobachtung von aktuellen Schiffsmanövern möglich, wie im Beispiel \ref{Schleppmanöver} das Schleppen eines Frachtschiffes durch zwei Schleppschiffe. Durch die hohe Frequenz der Positions-Meldungen bei fahrenden Schiffe und mithilfe der eingebauten Animation erlebt der Betrachter die Szene wie in einem Animationsfilm.\\

Mit dieser Implementierung wurde auch die wichtige nicht funktionale Anforderung nach einer zeitnahen Umsetzung erfüllt. Der Prototyp wird auf github als privates Repository gehostet und konnte nach der Übergabe (Ende Januar) vom Unternehmen vesseltracker für Weiterentwicklungen der Webanwendung und Kundenprojekte verwendet werden. 
Desweiteren ist die gesamte Anwendung mit open source Produkten entwickelt worden und verwendet das von vom Unternehmen gehostete Kartenmaterial.\newline
Die Anwendung wurde auf den unten angegebenen Browserclients in den angegebenen Versionen positiv auf Funktionalität getestet und unterstützt damit die gängisten Browser in den einschlägigen Versionen \footnote{http://www.browser-statistik.de/statistiken/versionen/}:
\begin{itemize}
\item Firefox Version 15.0.1 und 20.0
\item Google Chrome 26.0
\item Internet Explorer Version 9.0 und 10.0, und Version 7 und 8 im Kompatibilitätsmodus von IE 10
\item Safari 6.0.4
\end {itemize}

Die Latenzzeit zwischen dem Empfang der AIS-Positions-Meldung durch den Rohdatenserver und der Propagierung derselben Position auf der Karte sollte unterhalb 500 msec liegen. 
\begin {wraptable}{r}{3in}
\begin{center}
\begin{tabular}{| l|r|}\hline
& [msec]\\\hline
Mittelwert & 237 \\
Maximum & 701\\
Minimum & 23\\\hline
\end{tabular}
\caption[Latenzzeit von Positionsmeldungen in der RealtimeAnwendung]{Latenzzeit von Positionsmeldungen in der Realtime-Anwendung}
\label{Latenzeit_RealtimeApp}
\end{center}
 \end {wraptable}
Untersucht wurde die Latenzzeit anhand der ‘time\_received’, die vom Rohdatenserver jeder AIS-Message als Zeitstempel beim Empfang hinzugefügt wird. Auf dem Rohdatenserver läuft ein ntp-Daemon zur Zeitsynchronisation. Ebenso ein Daemon wurde auf dem Client, auf dem der Browser läuft gestartet und ein Zeitstempel genommen, nachdem das Schiff mit der neuen Position auf die Karte gerendert wurde. Weil die Latenzzeit direkt abhängig ist von der Anzahl der Meldungen, die pro Zeit vom Client emfangen werden, wurde als Referenz eine Ansicht des Hamburger Hafens gewählt in Zoomstufe 12, der niedrigsten Zoomstufe, in der noch Schiffe jeder Geschwindigkeit angezeigt werden. Die Ergebnisse (Tabelle \ref{Latenzzeit}) liegen mit dem Mittelwert gut innerhalb der geforderten Geschwindigkeit.

Die Anzahl der Verbindungen, die der Server gleichzeitig bedienen kann, ist mit einem node.js-Script getestet worden, das alle 500 ms eine neue Clientverbindung erstellt, bis 750 Clients verbunden sind. Der Aufbau der Verbindung geschah auch mit steigender Verbindungsanzahl zuverlässig, jedoch ist in der Abbildung erkennbar, dass die Anzahl der Fälle zunimmt, in denen ein Client lange auf Antwort warten muss.
\\Eine Skalierung der Serveranwendung ist seitens des socket.io-Servers kein Problem. Statt einen einzigen Worker-Prozess zu generieren, können auch mehrere Worker-Prozesse parallel gestartet werden, die alle dieselbe mongo-Datenbank-Collection abfragen und sich bei derselben redis-Datenbank im Channel ‘positionUpdate’ registrieren können. Allerdings muss sichergestellt sein, z.B.  über unterschiedliche Ports oder unterschiedliche (virtuelle) Server, dass ein verbundener Client mit jedem neuen Request auf demselben Worker-Prozess landet.

\begin{figure}[H]
\begin{minipage}[hbt]{3in}
	\centering
	\includegraphics[width=2.5in]{images/stresstest300.png}
	\label{Stresstest300}
\end{minipage}
\hfill
\begin{minipage}[hbt]{3in}
	\centering
	\includegraphics[width=2.5in]{images/stresstest.png}
	\label{Stresstest}
\end{minipage}
\caption{Anwortzeiten des socket.io-Websocket-Servers in Abhängigkeit von der Anzahl verbundener Clients}
\end{figure}
%-------------------------------------------------------------------------------------------------------------------------------------------
\section{Vergleichende Evaluation der Javascript- und der Dart-Anwendung}
Die in Google Dart geschriebene Client-Anwendung soll nun mit dem in Javascript programmierten Client verglichen werden. Die in den Abschnitten  \ref{Strategie-Korrektur} und \ref{Vergleichbarkeit} begründete zusätzliche Implementierung eines HTML5-kompatiblen node.js-Websocket-Servers muss nun in einem ersten Schritt mit dem node.js-socket.io-Server verglichen werden. Für diesen und alle folgenden Tests wurde auf dem Rohdatenserver ein Port eingerichtet, der nur die Daten von drei AIS-Antennen (Hamburg, Wedel, Geesthacht) ausgibt. Dies war notwendig, um den als Server verwendeten Arbeitsplatzrechner (MHz und 1 GB Arbeitsspeicher) nicht zu überlasten. Dadurch entstünde eine zusätzliche Latenzzeit wegen des kontinuierlichen Anwachsens der Messagequeues auf dem Server, die die Ergebnisse verzerrt.
%-------------------------------------------------------------------------------------------------------------------------------------------
\subsection{Node.js - socket.io-Socket-Server vs. node.js-Websocket-Server}\label{socket.io- vs html5-Server}
\subsubsection{Implementierungsaufwand}
Im Implementierungsaufwand unterscheiden sich beide Server- und Client-Anwendungen kaum (Anzahl zeilen code).  Einige Features der socket.io-Bibliotheken (z.B. die Clientverwaltung, Parameter für ‘Production’ und ‘Development’-Umgebung bezüglich Log-Leveln, Client-Minifikation oder Client-Zip) sind praktisch und müssten in der Alternativimplementierung für den Einsatz in einer produktiven Umgebung anderweitig gelöst werden. Durch die in socket.io eingeführten Events vermindert sich der Kommunikationsaufwand zwischen Server und Client geringfügig, was in diesem Fall einer datenintensiven Anwendung mit großen Datenmengen pro Nachricht wenig zu Buche schlägt.
\subsection{Latenzzeit}
Die Leistungsfähigkeit beider Implementierungen wird verglichen, indem wieder wie in Tabelle \ref{Latenzeit_RealtimeApp} die Zeit gemessen wird, die eine Positionsmeldung braucht für den Weg vom Rohdatenserver bis zur Präsentation auf der Karte. Um eine ähnliche Situation in beiden Szenarien abzubilden, wurde jeweils eine bestimmte Position in Hafen Hamburg angesteuert und ein Timer in die Client-Anwendung integriert, der jeweils nach einer Minute um eine Stufe herauszoomt. Da ab Zoomlevel 11 und kleiner nur noch Schiffe mit einer jeweils definierten Mindestgeschwindigkeit angezeigt werden, nahm die Anzahl empfangener Schiffe von dieser Zoomstufe an wieder ab. Zur Auswertung wird auf dem Client ein LogFile geschrieben, das pro Nachricht, deren ‘time\_received’ und einen aktuellen Zeitstempel schreibt.  Die Differenz wird als Latenzzeit interpretiert. Anschließend wird über das Logfile berechnet, wieviele Positionsmeldungen in einer Minute an den Client gesendet wurden. Darüber ist es möglich, die Latenzzeit gegen die Anzahl empfangener Positionsmeldungen pro Minute darzustellen, wie in Abbildung \ref{Latenzzeit socket.io} und \ref{Latenzzeit HTML5} zu sehen.
\begin {figure}[H]
\begin{center}
  \includegraphics[width=4.5in]{images/latency_timeReceived_socket_io.png}
\end{center}
\caption{socket.io-Websocket-Server: Latenzzeit der Positionsmeldungen und Anzahl empfangener Schiffe}
\label {Latenzzeit socket.io}
\end {figure}

\begin {figure}[H]
\begin{center}
  \includegraphics[width=4.5in]{images/latency_timeReceived_HTML5.png}
\end{center}
\caption{HTML5-Websocket-Server: Latenzzeit der Positionsmeldungen und Anzahl empfangener Schiffe}
\label {Latenzzeit HTML5}
\end {figure}
Es ist offensichtlich, dass die Geschwindigkeit in der Darstellung von der Anzahl empfangener Nachrichten linear abhängt.
Darüber hinaus ist zu erkennen, dass beide Implementierungen ihre Aufgabe in ähnlicher Geschwindigkeit erledigen. 

\subsection{Browserunterstützung}
Für die  node.js-HTML5-Websocket-Anwendung wird wie für die socket.io-Websocket-Anwendung in Abschnitt \ref{Evaluation der Anwendung} die Unterstützung durch die gängigsten Browser getestet. Die Ergebnisse sind in Tabelle \ref{Browserclients-Vergleich} gegenübergestellt.

\begin{table}
\begin{tabular}{ l|l|c|c}
Browser&Version&socket.io-Anw.&HTML5-Anw.\\\hline
Firefox& 20.0&websocket&websocket \\
Google Chrome & 26.0 &websocket&websocket\\
Internet Explorer&9& Fallback auf Flashsocket&nicht unterstützt\\
Internet Explorer&10& websocket&websocket\\
Safari&6.0.4 & websocket&websocket\\
\end{tabular}
\caption[Unterstützung von Browserclients im Vergleich socket.io vs. HTML5]{Unterstützung von Browserclients im Vergleich socket.io vs. HTML5}
\label{Browserclients-Vergleich}
\end{table}

\section{Javascript-Client vs. Dart-Client} 
\subsection{Implementierungsaufwand}
Beim Programmieren bietet Dart nach einer kurzen Einarbeitung einige Erleichterungen gegenüber der Javascript-Programmierung. Besonders ist das objektorientierte Programmieren in Dart sehr viel intuitiver möglich als in Javascript, wie leicht in Abbildung \ref{} zu erkennen. Javascript bietet zwar zahlreiche Lösungen, um Objektorientiertheit herzustellen über Funktionen als Objekte oder Konstrukte wie das Revealing Module Pattern, das aber verlangt javascript-spezifisches Hintergrundwissen und schafft keine vergleichbare Struktur.
Die Dart-Implementierung der Anwendung wird an dem Punkt etwas aufwendiger, wo über das Paket js-interop die Javascript-Dateien integriert werden und Proxies und Callback-Funktionen geschrieben werden müssen zur Kommunikation zwischen dem Javascript- und dem Dart-Namensraum.
Mit dem dart2js-Compiler ließ sich der Dart-Client-Code zu Javascript kompilieren und war dann auch auf Browsern ohne Dart VM ausführbar.
\subsection{Performance}
Um die unterschiedliche Performance des Javascript-Client, des eigentlichen Dart-Clients und des zu Javascript kompilierten Dart-Clients zu vergleichen, wurde der VesselInBounds-Event genutzt. Gemessen wurde die Zeit, die benötigt wird, um nach Empfang eines VesselInBounds-Events alle in der Message enthaltenen Schiffe auf die Karte zu rendern. Verglichen wurden die Clients in den Browsern Dartium, Chrome und Firefox.
\begin{itemize}
\item Google Dartium interpretiert den originären Dart-Client mit der Dart VM. Beim Aufruf des Javascript-Clients wird der Javascript-Code mit der V8-Javascript-Engine interpretiert.
\item Google Chrome  mit der V8-Javascript-Engine führt beim Aufruf des Dart-Clients den zu javascript kompilierten Dart-Code aus und beim Aufruf des Javascript-Clients den originären Javascript-Code.
\item ebenso führt Firefox mit der SpiderMonkey Javascript Engine den zu Javascript kompilierten Dart-Client-Code, bzw. den originären Javascript-Client-Code aus.
\end {itemize}


\begin {figure}[H]
\begin{center}
  \includegraphics[height=2.3in]{images/Dartium.png}
\end{center}
 \caption{Dauer des Renders in Dartium}
\end {figure}


\begin {figure}[H]
\begin{center}
  \includegraphics[height=2.3in]{images/Chrome.png}
\end{center}
 \caption{Dauer des Renders in Chrome}
\end {figure}


\begin {figure}[H]
\begin{center}
  \includegraphics[height=2.3in]{images/Firefox.png}
\end{center}
 \caption{Dauer des Renders in Firefox}
\end {figure}

Insgesamt ist erkennbar, dass der Javascript-Code schneller ausgeführt wird als der Dart-Code. Am schnellsten führt die V8-Javascript-Engine in Dartium und Chrome den originären Javascript-Code aus. Firefox mit der Spidermonkey-Javascript-Engine benötigt deutlich länger. Auffällig ist, dass der Dart-Code von der Dart VM in Dartium langsamer ausgeführt wird als der zu Javascript kompilierte Code in Chrome. 



\subsection{Browserunterstützung}
\subsubsection{Dartium}

\subsubsection{Firefox, Chrome, IE, Safari}

Der dart-Client kompiliert den in Dart geschriebenen Code zu Javascript.

Dabei traten Fehler auf, die unter Dartium (also im originalen Dart-Code) nicht auftraten.
1. Wird innerhalb des Javascript-Scopes eine Methode auf einen javascript-Proxy (hier \_map) aufgerufen und ein proxy wird zurückgegeben, dann ist es nicht möglich auf diesen Proxy, der in diesem Fall vom Typ LatLngBounds sein müsste, eine Methode der Klasse LatLngBounds aufzurufen. => TypeError: t1.get\$\_map(...).getBounds\$0(...).getSouthWest\$0 is not a function

dart-client: web/leaflet\_maps.dart

  List getBounds(){
    var south, west, north, east;
    js.scoped((){
    south= \_map.getBounds().getSouthWest().lng;
        west = \_map.getBounds().getSouthWest().lat;
        north = \_map.getBounds().getNorthEast().lng;
        east = \_map.getBounds().getNorthEast().lat;
 });
return [west, south, east, north];
    
In diesem Fall wird einfach als work-Around eine andere Methode verwendet (getBBoxString), die einen String mit den Bounds zurückgibt. Aus den Teilen dieses Strings werden mit der Methode parse(string) der Klasse double die Werte der Eckpunkte der Bounds generiert.

String getBounds(){
    String bBox;
    js.scoped((){
      bBox = \_map.getBounds().toBBoxString();
    });
    return bBox;
  }

 Weil dadurch der message-Parameter 'bounds' kein number-Array, sondern ein String ist, muss im html5-Server der String einmal zum Float geparst werden.



2. Ein Feld ("IMO") wird auf null und auf > 0 geprüft.


\chapter{Fazit}\label{Fazit}
 \section{Ergebnisse }
\subsection{Die Realtime-Applikation}


Features, die in die Anwendung integriert werden sollten:
\begin{itemize}
\item eine aufklappbare Liste, in die der Nutzer favorisierte Häfen speichern kann, die dann mit einem Klick angesteuert werden können.
\item eine Legende mit einer Erläuterung der farbigen Schiffssymbole
\item zur weiteren Optimierung der Anwendung, sollte die Möglichkeit, die Animation der Richtungsdreiecke und Schiffspolygon über css3-Transition-Funktionen zu realisieren, unbedingt weiterverfolgt werden
\end{itemize}

\subsection{Vergleich Javascript und Google Dart}

Wie in den Abbildung zur Struktur der Anwendung erkennbar, ist mit Dart sehr viel einfacher Objektorientiertheit herzustellen. Dadurch wird die Anwendung einfacher zu verstehen und entsprechend einfacher zu entwerfen, zu schreiben und zu warten.
Das Erlernen von Dart ist für Umsteiger von Javascript relativ einfach möglich. Von den vielen Features und Möglichkeiten, die Dart bietet, kommt in der Client-Anwendung zwar nur ein kleiner Teil zum Tragen. Das Ziel war hier nicht das Ausschöpfen der Möglichkeiten von Google Dart, sondern die Realisierung eines kleinen Projektes alternativ in Google Dart.
Sehr hilfreich war der DartEditor als Entwicklungsumgebung, dessen Compiler bei Syntax-Fehlern aussagekräftige Warnungen ausgibt. 
Google Dart befindet sich noch in der Entwicklungsphase, so dass gelegentlich nach den ca. wöchentlichen Versions-Updates von Dart die Anwendung unter der neuen Version nicht mehr lauffähig ist und angepasst werden muss. Belohnt wird der Aufwand sozusagen mit einer kontinuierlichen Verbesserung der Performance im Laufe des letzten halben Jahres.
Das Einbinden existierender Javascript-Bibliotheken mit js-interop erweitert die Nutzungsmöglichen von Dart ganz entscheidend. Das Arbeiten mit zwei unterschiedlichen Scopes (Javascript und Dart) ist aber am Anfang sehr fehleranfällig. Und die Fehler sind schwierig zu debuggen, weil die Debugger (hier Firebug und der Debugger im DartEditor) nicht über die Grenzen des Namensraumes wechseln können.

\section{Ausblick}
Legende
-Satellitendaten in die Anwendung einbinden
%http://en.wikipedia.org/wiki/Comparison_of_layout_engines_%28ECMAScript%29#ECMAScript_version_support


%--------------------------------------------------------------------------------------------------------------------------------------------


\bibliographystyle{alphadin_martin}
\bibliography{literatur}


%---------------------------------------------------------------------------------------------------------------------------------------------
\chapter*{Erklärung}

Hiermit versichere ich, dass ich die vorliegende Arbeit selbstständig verfasst und keine anderen als die angegebenen Quellen und Hilfsmittel benutzt habe, dass alle Stellen der Arbeit, die wörtlich oder sinngemäß aus anderen Quellen übernommen wurden, als solche kenntlich gemacht und dass die Arbeit in gleicher oder ähnlicher Form noch keiner Prüfungsbehörde vorgelegt wurde.

\vspace{3cm}
Ort, Datum \hspace{5cm} Unterschrift\\

\end{document}