
\chapter{Implementierungen}\label{s.Implementierungen}

\section{Strategie bei der Vorgehensweise}\label{Strategie bei der Vorgehensweise}
Zunächst wird eine Implementierung gewählt, die die besten Chancen hat, alle Anforderungen zu erfüllen. Diese steht im Zeitplan ganz vorne, um der Anforderung von Unternehmensseite nach einer zeitnahen Umsetzung und Auslieferung zu entsprechen.
Dies ist eine Lösung in Javascript mit dem node.js-Framework und dem socket.io-Websocket (Abschnitt \ref{socket.io-Server} und \ref{socket.io-Client}). Node.js-Serveranwendungen werden schon länger mit guten Ergebnissen in Netzwerken eingesetzt, besonders für Realtime-Anwendungen und vielen gleichzeitig verbunden Clients. Das socket.io-Paket wird genutzt, weil durch die Kapselung der verschiedenen Transportmechanismen die Bedienung einer maximalen Anzahl an Browser-Clients möglich ist, ohne den Implementierungsaufwand unverhältmismäßig zu erhöhen.
In einem zweiten Schritt wird eine vergleichbare Implementierung in Google Dart ausgeführt. Die Entwicklung von Dart befindet sich noch in der Beta-Phase. Der zweite Beta-Release fand im Dezember 2012 statt. Ein dritter Beta-Release ist angekündigt. Ein zeitnaher ausschließlicher Einsatz von Dart im Produktivsystem ist somit ausgeschlossen und diese Lösung ist als Investition in die Zukunft zu sehen. 
Der Vergleich beider Implementierungen (Javascript vs. Dart) ist deshalb nicht weniger interessant.   
\section{Notwendige Strategie-Korrekturen}
Der ursprüngliche Plan, sowohl Server als auch Client in Dart zu schreiben, musste korrigiert werden, weil mit dem Dart-Websocket-Server einige der grundlegenden Anforderungen nicht umzusetzen waren. Zum einen unterstützt Dart keine JSON-over-TCP -Kommunikation, wie sie für die Abfrage des JSON-Datenstroms vom Rohdatenserver erforderlich ist. Und zum anderen gab es noch keinen Redis-Client für Dart. Der publish/subscribe Mechanismus der Redis-Datenbank wird für die Verteilung der Positionsupdates benötigt.
Deshalb wird nur der Client in Dart implementiert (\ref{HTML5-Client in Dart}). Dadurch ergibt sich ein weiteres Problem: der socket.io-Websocket-Server entspricht nicht der HTML5-Websocket-API-Spezifikation und benötigt deshalb auf Clientseite zusätzliche Bibliotheken. Diese Bibliotheken stehen in Dart nicht zur Verfügung. Dart unterstützt Websocketverbindungen clientseitig mit dem Paket dart:html. Darin wird ein Websocket nach der HTML5-Websocket-API-Spezifikation erwartet.
Folglich muss neben dem socket.io-Server ein zweiter Server (in Javascript) implementiert werden, der eine Websocket-Verbindung nach der HTML5-Websocket-API-Spezifikation aufbaut (\ref{HTML5-Server}). Dies ist relativ einfach  möglich: in node.js kann hierfür das Modul websocket eingebunden werden.
\section{Das Problem der Vergleichbarkeit}
An dieser Stelle stellt sich die Frage, ob beide Lösungen direkt vergleichbar sind. Mögliche Unterschiede zwischen den node.js-Servern (socket.io vs. HTML5) würden in das Ergebnis des Vergleichs zwischen den Clients (Dart vs Javascript) einfliessen. Deshalb wird der Javascript-Client noch einmal mit dem HTML5-Websocket implementiert, der dann auf denselben HTML5-Websocket-Server zugreift wie der Dart-Client. 
Es werden also zwei Vergleiche durchgeführt: 
\begin{itemize}
\item In Javascript wird der socket.io-Websocket gegen den HTML5-Websocket getestet.
\item Unter Verwendung des HTML5-Websockets wird der Javascript-Client gegen den Dart-Client getestet
\end{itemize}
Tabelle\ref{tab:uebersicht} zeigt eine Übersicht über die Implementierungen.
Auf diese Weise wird die präferierte Lösung in node.js mit socket.io (S1) nicht unmittelbar sondern mittelbar über die Javascript-Lösung mit HTML5-Websocket (H1) gegen die Implementierung mit einem Google Dart Client (H2) getestet.
\renewcommand{\arraystretch}{1.2}

\begin{table}[!hbt]\vspace{1ex}\centering
\begin{tabular}{| l| m{3.5cm}||c|c|}\cline{3-4}

\multicolumn{2}{c||}{}&\multicolumn{2}{c|}{HTTP-Client}\\\cline{3-4}
\multicolumn{2}{c||}{}& Javascript& Google Dart\\\hline\hline
\multirow{2}*{\rotatebox{90}{HTTP-Server}}& socket.io-Websocket-Server  (\ref{socket.io-Server}) &  socket.io-Client  (\ref{socket.io-Client})& \includegraphics[width=0.2in]{images/x_red.jpeg}\\\cline{2-4}
&HTML5-Websocket-Server (\ref{HTML5-Server}) & HTML5-Client (\ref{HTML5-Client in Javascript}) & HTML5-Client  (\ref{HTML5-Client in Dart})\\\hline
\multicolumn{4}{c}{}\\
\end{tabular}
\caption[Übersicht über Server-und Clientimplementierungen]
{Übersicht über Server-und Clientimplementierungen\\}
\vspace{2ex}
\label{tab:uebersicht}
\end{table}
\newpage

\section{Beschreibung der ausgeführten Implementierungen}
In der ersten Implementierung werden Lösungen entwickelt für die in den Anforderungen beschriebenen Aufgaben. In allen weiteren Implementierungen werden diese Lösungen möglichst übernommen und nur, wo das nicht möglich ist wird eine Alternative entwickelt.

\subsection{socket.io-Server}\label{socket.io-Server}
Die zu entwickelnde Serveranwendung hat grundsätzlich zwei Aufgaben: Daten über eine JSON-over-TCP-Verbindung vom Rohdatenserver abzurufen und einen Websocket zu unterhalten, der die Daten an Websocket-Clients weitergibt.
Weil Node.js singlethreaded ist (vgl. \cite{teixeira}), würde der Server beide Aufgaben in einem einzigen Prozess bearbeiten. Um das Potential an Parallelverabeitung eines Dualcore oder Multicore-Servers zu nutzen, ist des daher sinnvoll, mindestens zwei Prozesse zu generieren. Dazu wurde das node.js-Modul child\_process genutzt. Die Start-Datei master.js generiert damit zuerst einen Prozess, der den AIS-Daten-Client (aisData-client.js) abzweigt, um Daten vom Rohdaten-Server abzufragen und anschließend einen worker-Prozess (worker.js), um einen Websocket -Server für Client-Verbindungen zur Verfügung zu stellen (siehe \ref{lst:master.js}).
\begin{lstlisting}[caption=Generierung von Kindprozessen in master.js, firstnumber=16, label=master.js]
/* AIS-Client - Process*/
  child.fork(path.join(__dirname, 'aisData-client.js'));

/*worker- Process*/
  child.fork(path.join(__dirname, 'worker.js'));
\end{lstlisting}
Bei der Weitergabe der Daten durch den worker-Prozess sind zwei Fälle zu unterscheiden:
\begin{itemize}
\item ein Client verbindet sich neu oder ändert den Kartenausschnitt. In diesem Fall (Vessel-in-Bounds Request) sind die Schiffs-und Positionsdaten aller im Bereich (Bounds) befindlichen Schiffe an den Client zu senden (Kapitel \ref(vessel-in-Bounds Request).
\item ein Schiff sendet ein Positions-Update, das an alle Clients verteilt werden muss, deren Kartenausschnitt die betreffende Schiffsposition enthält. Dieses Ereignis wird im Folgenden Vessel-Position-Update genannt (Kapitel \ref{Vessel-Position-Update}).
\end{itemize}
\subsubsection{Vessels-in-Bounds-Request}\label{Vessels-in-Bounds-Request}
Der Vessels-in-Bounds-Request macht eine Zwischenspeicherung der Daten unumgänglich. Wegen der großen Anzahl gleichzeitig empfangener Schiffe (weltweit ca. 60.000) und der Notwendigkeit, einen geographischen Index zu verwenden, wird einer persistenten gegenüber einer transienten Speicherung der Vorzug gegeben. 
\\Für die Persistierung wird hier MongoDB verwendet, weil MongoDB als NoSQL-Datenbank mit geringem Overhead schnelle Antwortzeiten und außerdem einen geographischen Index bietet. Der Serverprozess in aisData-client.js schreibt die Daten (siehe listing \ref{write in Mongo}). Die MMSI eines Schiffes ist eindeutig und wird als unique key verwendet (siehe Abschnitt \ref{Statische Schiffsdaten}). Über die Option upsert:true wird der Mongo Datenbank mitgeteilt, dass entweder ein insert-Befehl oder, falls die mmsi bereits in der MongoDB-Collection vorhanden ist, ein update-Befehl auf das entsprechende set ausgeführt werden soll. 
\begin{lstlisting}[caption=Schreiben in die Datenbank in aisData-client.js, firstnumber=321, label=write in Mongo]
vesselsCollection.update(
  { mmsi: obj.mmsi },
  { $set: obj },
  { safe: false, upsert: true }
  );
\end{lstlisting}
Der Geo-Index ist zwingend erforderlich, damit nicht jede Anfrage des Servers an die Datenbank sämtliche Datensätze durchlaufen muss, um die Schiffe in einem bestimmten geographischen Ausschnitt zu finden. Aufbau und Unterhalt des Geo-Indexes findet im aisData-client-Prozess statt, der schreibend auf die Datenbank zugreift.
\begin{lstlisting}[caption=Aufbau des Geo-Indexes in aisData-client.js]
  vesselsCollection.ensureIndex({ pos: "2d", sog: 1, time_received: 1 }, function(err, result) {... });
  \end{lstlisting}
Dabei handelt es sich um einen zusammengesetzten Index, weil neben der Geo-Position auch die Geschwindigkeit und der Zeitpunkt des Empfangs der Nachricht Filterkriterien sind, wenn der zweite Prozess (worker.js) lesend auf die Datenbank zugreift. In Listing \ref{lst:query Mongo} ist zu sehen, wie der Prozess worker.js mit den vom Client in einem Vessels-in-Bounds-Request übermittelten Geo-Daten die MongoDb anfragt.
  \begin{lstlisting} [caption=Vessel-in-Bounds-query in worker.js, label=lst:query Mongo]
  var vesselCursor = vesselsCollection.find({
    pos: { $within: { $box: [ [bounds._southWest.lng,bounds._southWest.lat], [bounds._northEast.lng,bounds._northEast.lat] ] } },
    time_received: { $gt: (new Date() - 10 * 60 * 1000) },
    sog: { $exists:true },
    sog: { $gt: zoomSpeedArray[zoom]},
    sog: {$ne: 102.3}
  });
  vesselCursor.toArray(function(err, vesselData)  {
   client.sendUTF(JSON.stringify({ type: 'vesselsInBoundsEvent', vessels: vesselData}));
});
\end{lstlisting}

\subsubsection{Vessel-Position-Update}\label{Vessel-Position-Update}
Für die Kommunikation eines Vessel-Position-Updates (AIS-Nachrichtentyp 1-3) zwischen dem aisData-client.js-Prozess und dem worker.js-Prozess wird der publish/subscribe-Mechnismus einer Redis-Datenbank genutzt. Der aisData-client.js-Prozess publiziert jedes Positions-Update auf dem Kanal ‘vessel-Pos’ der Redis-Datenbank. Der worker.js-Prozess meldet sich als subscriber am Kanal ‘vessel-Pos’ der Redis-Datenbank an und wird so bei jedem Positions-Update benachrichtigt.
Um diese Nachricht an die betroffenen Websocket-Clients weiterzuleiten, ist eine serverseitige Verwaltung der Clients notwendig. Die Serveranwendung muss bei jeder Positionsmeldung wissen, welche Clients benachrichtigt werden müssen. Die Client-Verwaltung ist ein Feature des socket.io-Paketes. Für jeden Client wird bei der Registrierung zusätzlich das Zoomlevel der Karte und die Koordinaten gespeichert.
\begin{lstlisting}[caption= Speichern der übermittelten Client-Daten in worker.js, label=Speichern der übermittelten Client-Daten in worker.js]
io.sockets.on('connection', function(client) {
      log(' Connection from client accepted.');
      client.on('register', function(bounds, zoom) {
      client.set('zoom', zoom);
      client.set('bounds', bounds, function() {
        getVesselsInBounds(client, bounds, zoom);
      });
    });
    client.on('unregister', function() {
      client.del('bounds');
      client.del('zoom');
    });
  });
  \end{lstlisting}
  Bei jedem Vessel-Position-Update, das der worker.js-Prozess empfängt, geht er die Liste der Clients durch und benachrichtigt diejenigen, in deren Bereich das Positions-Update fällt.
\begin{lstlisting}[caption= Weiterleitung von Positions-Updates an Websocket-Clients in worker.js, label= Weiterleitung von Positions-Updates an Websocket-Clients in worker.js]
 redisClient.on('message', function(channel, message) {
    if (channel == 'vesselPos')  {
      ...
      var json = JSON.parse(message);
      ...
      var clients = io.sockets.clients();
      var lon = json.pos[0];
      var lat = json.pos[1];
      var sog = json.sog/10;
      var cog = json.cog/10;
      clients.forEach(function(client) {
        client.get('bounds', function(err, bounds) {
          if (bounds != null && lon != null && lat != null) 
          {
            /* check, if Client-Connection is affected by Vessel-Position-Update */
            if (positionInBounds(lon, lat, bounds)) 
            {
              client.get('zoom', function(err, zoom) 
              {
                if(sog !=null && sog > (zoomSpeedArray[zoom]) && sog != 102.3)
                {
                  client.emit('vesselPosEvent', message);
                }
          ...
  });
\end{lstlisting}

\begin{figure}[H]
  \centering
  \includegraphics[width=6in]{images/ais-socketio.png}
  \caption[Übersicht Javascript-Files]{Übersicht Javascript-Files}
  \label{fig:Übersicht Javascript-Files}
\end{figure}
\subsection{socket.io-Client}\label{socket.io-Client}
Der socket.io-Client hat folgende Aufgaben.
\begin{enumerate}
\item Es ist eine html-Seite mit den benötigten Bereichen für die Karte zu erstellen.
\item URL-Parameter sollen optional übergeben werden könnenn.
\item Eine Karte mit Navigations- und Infobereichen ist in den Kartenbereich zu rendern.
\item Zum Socket.io-Websocket-Server soll eine Verbindung aufgebaut werden
\begin{itemize}
\item bei Positionsänderungen auf der Karte wird eine register-Nachricht gesendet
\item Vessel-In-Bounds- und Vessel-Position-Events können empfangen werden
\end{itemize}
\item Aus den gesendeten Daten sind geeignete Objekte zu erstellen und zu speichern.
\item Die Objekte sind als Features auf die Karte zu rendern
\item Objekte, die den Status ‘Moving’ haben, sind entsprechend ihrer Geschwindigkeit zu animieren.
\end{enumerate}
Punkt 1 geschieht in der Datei ais-socket.io.html. Dort werden auch die benötigten javascript- und css-Dateien eingebunden. Falls Parameter übergeben werden für Zoomlevel und Bereich werden sie in der javascript-Datei ais-socket.io.js mit der Funktion getParam(name) aufgegriffen.
Als Kartenwerk werden OpenStreetMap-Karten benutzt (siehe Kapitel \ref{OpenStreetMap}). Als Javascript-Bibliothek zur Darstellung der Schiffe auf der Karte fiel die Wahl auf die Leaflet-Bibliotheken. Zwar stützt sich die Cockpit-Anwendug der Vesseltracker-GmbH auf die älteren OpenLayers-Bibliotheken, diese sind jedoch im Vergleich sehr viel sperriger in der Nutzung und werden inzwischen weniger aktiv von der Community weiterentwickelt und verbessert.
Die Karte wird als Javascript-Objekt in der Datei LeafletMap.js realisiert mithilfe des ‘Revealing Module Pattern’. Damit läßt sich die API von der internen Implementierung der Karte trennen. Nur die zurückgegebenen Funktionen bilden die öffentliche Schnittstelle.

\begin{lstlisting}[caption= ‘Revealing Module Pattern’ in LeafletMap.js, label=LeafletMap.js]
var LM = function(){
    var map, featureLayer, tileLayer, zoom, socket, boundsTimeout;
	
    function init(divName, options){...}
    function addOSMLayerToMap(){...}
    function changeRegistration(){...}
    function getMap(){...}
    function getZoom(){...}
    function paintVessel(vessel){...}
    function removePopups(){...}
    function clearFeature(vessel){...}
    
    return {
        init: init,
        getMap: getMap,
        getZoom: getZoom,
        paintVessel: paintVessel,
        clearFeature: clearFeature
    }
}();
\end{lstlisting}

Mit den folgenden Befehlszeilen wird die Websocket-Verbindung hergestellt. Zum Empfang der Nachrichten des Websocket-Servers (siehe Kapitel \ref{socket.io-Server}) genügen zwei Listener: 
\begin{lstlisting}[caption=Client-Seite der socket.io-Websocket-Verbindung in ais-socket.io.js,  label=ais-socket.io.js]
var socket = io.connect('http://127.0.0.1:8090');

socket.on('vesselsInBoundsEvent', function (data) {...}
socket.on('vesselPosEvent', function (data) {...}
\end{lstlisting}

Bezüglich der an den Websocket-Server zu sendenden register-Nachricht ergab sich durch die Kapselung der LeafletMap als revealing Module eine Besonderheit: der moveEnd-Event, der die neue Registrierung triggern soll, befindet sich im LeafletMap-Objekt. Als Lösung wird diese Aufgabe dem LeafletMap-Objekt übertragen, weswegen bei der init-Funktion der socket mit übergeben wird. Diese Lösung ist unproblematisch, weil beim Verlust der socket-Verbindung ohnehin ein Reload der Seite stattfinden muss.
Um geeignete Objekte aus den Ais-Messages zu erstellen (Punkt 5), wird in der Datei Vessel.js eine Konstruktor-Funktion zur Verfügung gestellt, mit der Instanzen von Vessel-Objekten generiert werden können. Diese Instanzen werden in einem assoziativen Array, also einem Objekt, namens ‘vessels’ gespeichert. Beim Empfang eines Vessel-Position-Events kann mit vessels[mmsi] nach dem passenden vessel-Objekt zum Update gesucht werden.
Schließlich sind die Vessel-Objekte auf die Karte zu rendern. Dazu wird in Vessel.js eine asynchrone Funktion genutzt (siehe Listing \ref{vessel.createMapObjects}), mit der zuerst je nach Status (moving / not moving) und zoomlevel unterschiedliche Features für ein Schiff erstellt werden (Polygon, Triangle, Speedvector, Circle). Anschließend wird das vessel-Objekt auf die Karte gerendert und das Objekt mit den Features gespeichert. Das ist notwendig, um die Features bei Vessel-Position-Updates von der Karte zu entfernen, bevor sie an der neuen Position dargestellt werden.
\begin{lstlisting}[caption=Aufruf der public function createMapObjects des Vessel-Objekts in ais-socket.io.js, label=vessel.createMapObjects]
vessel.createMapObjects(LM.getZoom(), function(){
              LM.paintVessel(vessel);
              vessels[vessel.mmsi] = vessel;
            });
\end{lstlisting}

Für díe letzte Aufgabe, die Animation, wird das Polygon-Objekt des Leaflet-Frameworks erweitert zur Klasse L.AnimatedPolygon. Ein Polygon in Leaflet ist eine Polyline, die mehrere Punkte auf der Karte verbindet. Die Animation eines Punktes entlang einer Linie mit einer bestimmten Geschwindigkeit wurde aus dem Leaflet-Plugin L.AnimatedMarker \footnote{\label{foot:2}https://github.com/openplans/Leaflet.AnimatedMarker} übernommen. Die dort präferierte css3-Transition zur Animation konnte aber nicht verwendet werden, weil dazu ein Objekt im DOM-Baum verwendet werden muss. Das von Leaflet für ein Polygon erstellte DOM-Objekt (svn-Graphik) ist aber durch die Leaflet-Bibliothek gekapselt, so dass ein Zugriff von außen nicht vorgesehen ist.  \\

Ein Schiffspolygon wird berechnet aus der Schiffsposition (Positionsangabe in der AIS-Nachricht) und aus der relativen Position des AIS-Transceivers an Bord (Abstand zum Bug, zum Heck, nach Backbord und nach Steuerbord). Um zu wissen, in welche Richtung der Bug eines Schiffes zeigt, wird die AIS-Angabe zur Fahrtrichtung (cog = Course over Ground) verwendet. Aus dieser Richtungsangabe und der übermittelten Geschwindigkeit (sog = Speed over Ground) wird ein Speedvector berechnet, der als Linie auf die Karte gezeichnet wird. Bei der Erstellung des AnimatedPolygon-Objektes wird dieser Speedvector mit übergeben, um bei der Animation des Polygons die Position des Schiffes entlang des Speedvektors zu verschieben. Nach jedem Animationsschritt wird das Polygon neu berechnet und gezeichnet.


\subsection{HTML5-Server}\label{HTML5-Server}
Diese Server-Implementierung soll genau dieselbe Funktionalität besitzen wie die socket.io-Server-Implementierung (\ref{socket.io-Server}). Lediglich der Websocket socket.io wird durch einen node.js-Websocket nach der HTML5-Websocket-Spezifikation ausgetauscht \footnote{\label{foot:2}https://github.com/Worlize/WebSocket-Node} . Dazu wird in der Datei worker.js das entsprechende Paket (‘websocket’) eingebunden.
Einige Features des socket.io-Paketes müssen jetzt selbst organisiert werden: 
\begin{itemize}
\item die Clientverwaltung erfolgt in einem Array ‘clients’, in dem zu jeder Zeit alle verbundenen Clients mit ihren Attributen stehen. 
\item in der Websocket können keine eigenen Events definiert werden (siehe API-Dokumentation \footnote{\label{foot:2}https://github.com/Worlize/WebSocket-Node/wiki/Documentation}. Deshalb wird der message-Event genutzt und die aufzurufende Funktion wird in der message übermittelt.
\end{itemize}
\begin{lstlisting}[caption= vom Websocket-Server gesendete message, label=websocket-message]
message origin=ws://127.0.0.1:8090, data={"type":"vesselsInBoundsEvent","vessels":[{"_id":"50d9fdb4bcc2e678a9278c18","aisclient_id":57,"callsign":"OVYC2 ","cog":285,"dest":"HAMBURG ","dim_bow":70,"dim_port":10,"dim_starboard":4,"dim_stern":30,"draught":54,"imo":"9363170","mmsi":220515000,"msgid":1,"name":"RIKKE THERESA ","nav_status":0,"pos":[9.85185,53.54643333333333],"rot":0,"ship_type":80,"sog":10.3,"time_captured":1366733896000,"time_received":1366733855248,"true_heading":286}]}

\end{lstlisting}
\begin{lstlisting}[caption= vom socket.io-Server gesendete message, label=socket.io-message]
[{"_id":"50d9fdb4bcc2e678a9278c18","aisclient_id":57,"callsign":"OVYC2  ","cog":286,"dest":"HAMBURG             ","dim_bow":70,"dim_port":10,"dim_starboard":4,"dim_stern":30,"draught":54,"imo":"9363170","mmsi":220515000,"msgid":1,"name":"RIKKE THERESA       ","nav_status":0,"pos":[9.8375,53.54885],"rot":0,"ship_type":80,"sog":12.4,"time_captured":1366734056000,"time_received":1366734014715,"true_heading":288}]
\end{lstlisting}

\subsection{HTML5-Client in Javascript}\label{HTML5-Client in Javascript}
Die socket.io-Client-Implementierung und die HTML5-Client bieten die gleiche Funktionalität und arbeiten nur geringfügig unterschiedlich.
 \begin{itemize}
 \item wie inListing \ref{socket.io-message} und \ref{websocket-message} zu sehen, muss der HTML5-Client zuerst den message-type abfragen, um die Daten korrekt zuzuordnen.
 \item weil dem Leaflet-Map-Objekt die Websocket-Connection als Parameter übergeben wird, muss dafür der Aufbau der Websocket-Connection abgewartet werden. 
\end{itemize}

\subsection{HTML5-Client in Dart}\label{HTML5-Client in Dart}
Die Implementierung des HTML5-Client in Dart orientiert sich an der Implementierung des HTML5-Websocket-Clients in Javascript. Oberstes Ziel dabei ist es, dieselbe Funktionalität zu implementieren wie in \ref{HTML5-Client in Javascript} unter Ausnutzung der sprachspezifischen Vorteile. \\

Die Modularisierung, also die Verteilung der Objekte und Funktionalitäten auf Programmdateien, ist in der Dart-Client-Implementierung ähnlich gestaltet wie in Javascript (siehe Abb. \ref{fig:Übersicht Javascript-Files}). Die verwendeten Javascript-Dateien (Leaflet-Bibliothek, L.AnimatedPolygon, L.Control.Mouseposition) sind identisch mit denen in \ref{HTML5-Client in Javascript}. Um sie aus Dart heraus nutzen zu können, wird das Dart-Paket js-interop (\ref{js-interop}) eingebunden.

Die main-Funktion in ais-html5.dart startet die Anwendung. Der html5-Websocket wird initialisiert und anschließend (über die Callback-Funktion) die Karte. So kann der moveend-Event, der nach dem Rendern der Karte ausgelöst wird, über die changeRegistration-Funktion und den bereitgestellten websocket den ersten Vessel-In-Bounds-Request senden.



, die die beiden Dart-Libraries Vessel (Vessel.dart) und LMap (LeafletMap.dart) einbindet. Die Library Vessel stellt die Klasse Vessel für die Initialisierung der Vessel-Objekte zur Verfügung, die Library leaflet\_map stellt die abstrakte Klasse LeafletMap zur Verfügung mit ihrer konkreten Ausprägung OpenStreetMap 
